% Options for packages loaded elsewhere
\PassOptionsToPackage{unicode}{hyperref}
\PassOptionsToPackage{hyphens}{url}
%
\documentclass[
]{book}
\usepackage{amsmath,amssymb}
\usepackage{lmodern}
\usepackage{iftex}
\ifPDFTeX
  \usepackage[T1]{fontenc}
  \usepackage[utf8]{inputenc}
  \usepackage{textcomp} % provide euro and other symbols
\else % if luatex or xetex
  \usepackage{unicode-math}
  \defaultfontfeatures{Scale=MatchLowercase}
  \defaultfontfeatures[\rmfamily]{Ligatures=TeX,Scale=1}
\fi
% Use upquote if available, for straight quotes in verbatim environments
\IfFileExists{upquote.sty}{\usepackage{upquote}}{}
\IfFileExists{microtype.sty}{% use microtype if available
  \usepackage[]{microtype}
  \UseMicrotypeSet[protrusion]{basicmath} % disable protrusion for tt fonts
}{}
\makeatletter
\@ifundefined{KOMAClassName}{% if non-KOMA class
  \IfFileExists{parskip.sty}{%
    \usepackage{parskip}
  }{% else
    \setlength{\parindent}{0pt}
    \setlength{\parskip}{6pt plus 2pt minus 1pt}}
}{% if KOMA class
  \KOMAoptions{parskip=half}}
\makeatother
\usepackage{xcolor}
\usepackage{longtable,booktabs,array}
\usepackage{calc} % for calculating minipage widths
% Correct order of tables after \paragraph or \subparagraph
\usepackage{etoolbox}
\makeatletter
\patchcmd\longtable{\par}{\if@noskipsec\mbox{}\fi\par}{}{}
\makeatother
% Allow footnotes in longtable head/foot
\IfFileExists{footnotehyper.sty}{\usepackage{footnotehyper}}{\usepackage{footnote}}
\makesavenoteenv{longtable}
\usepackage{graphicx}
\makeatletter
\def\maxwidth{\ifdim\Gin@nat@width>\linewidth\linewidth\else\Gin@nat@width\fi}
\def\maxheight{\ifdim\Gin@nat@height>\textheight\textheight\else\Gin@nat@height\fi}
\makeatother
% Scale images if necessary, so that they will not overflow the page
% margins by default, and it is still possible to overwrite the defaults
% using explicit options in \includegraphics[width, height, ...]{}
\setkeys{Gin}{width=\maxwidth,height=\maxheight,keepaspectratio}
% Set default figure placement to htbp
\makeatletter
\def\fps@figure{htbp}
\makeatother
\setlength{\emergencystretch}{3em} % prevent overfull lines
\providecommand{\tightlist}{%
  \setlength{\itemsep}{0pt}\setlength{\parskip}{0pt}}
\setcounter{secnumdepth}{5}
\usepackage{booktabs}
\usepackage{amsthm}
\makeatletter
\def\thm@space@setup{%
  \thm@preskip=8pt plus 2pt minus 4pt
  \thm@postskip=\thm@preskip
}
\makeatother

\usepackage{tcolorbox}


\newtcolorbox{blackbox}{
  colback=black,
  coltext=white,
  colframe=black,
  boxsep=5pt,
  arc=4pt}
\newtcolorbox{bonus}{
  colback=blue!15,
  colframe=blue!15,
  coltext=black!80,
  boxsep=5pt,
  arc=4pt}
\newtcolorbox{reflect}{
  colback=green!5,
  colframe=green!5,
  coltext=black!80,
  boxsep=5pt,
  arc=4pt}
\newtcolorbox{assessment}{
  colback=blue!5,
  colframe=blue!5,
  coltext=black!80,
  boxsep=5pt,
  arc=4pt}
\newtcolorbox{progress}{
  colback=purple!10,
  colframe=purple!10,
  coltext=black!80,
  boxsep=5pt,
  arc=4pt}
\newtcolorbox{video}{
  colback=yellow!5,
  colframe=yellow!5,
  coltext=black!80,
  boxsep=5pt,
  arc=4pt}
\newtcolorbox{caution}{
  colback=red!5,
  colframe=red!5,
  coltext=black!80,
  boxsep=5pt,
  arc=4pt}
\newtcolorbox{feedback}{
  colback=black!5,
  colframe=black!5,
  coltext=black!80,
  boxsep=5pt,
  arc=4pt}
\ifLuaTeX
  \usepackage{selnolig}  % disable illegal ligatures
\fi
\usepackage[]{natbib}
\bibliographystyle{apalike}
\IfFileExists{bookmark.sty}{\usepackage{bookmark}}{\usepackage{hyperref}}
\IfFileExists{xurl.sty}{\usepackage{xurl}}{} % add URL line breaks if available
\urlstyle{same} % disable monospaced font for URLs
\hypersetup{
  pdftitle={Leadership 101},
  pdfauthor={Course Developed by Colin Madland \& Kelly Marjanovic},
  hidelinks,
  pdfcreator={LaTeX via pandoc}}

\title{Leadership 101}
\author{Course Developed by Colin Madland \& Kelly Marjanovic}
\date{Last updated Aug 2023}

\usepackage{amsthm}
\newtheorem{theorem}{Theorem}[chapter]
\newtheorem{lemma}{Lemma}[chapter]
\newtheorem{corollary}{Corollary}[chapter]
\newtheorem{proposition}{Proposition}[chapter]
\newtheorem{conjecture}{Conjecture}[chapter]
\theoremstyle{definition}
\newtheorem{definition}{Definition}[chapter]
\theoremstyle{definition}
\newtheorem{example}{Example}[chapter]
\theoremstyle{definition}
\newtheorem{exercise}{Exercise}[chapter]
\theoremstyle{definition}
\newtheorem{hypothesis}{Hypothesis}[chapter]
\theoremstyle{remark}
\newtheorem*{remark}{Remark}
\newtheorem*{solution}{Solution}
\begin{document}
\maketitle

{
\setcounter{tocdepth}{1}
\tableofcontents
}
\hypertarget{welcome}{%
\chapter*{Welcome}\label{welcome}}
\addcontentsline{toc}{chapter}{Welcome}

This is the course book for LDRS 101: Learning with Technology. This book is divided into 6 units of study to help you engage with the course learning outcomes and prepare for the course assessment.

On the page below you will find a summary of the course syllabus, as well as how to navigate this book. Please also refer the schedule in Moodle, as well as the Assessment section in Moodle for instructions on assignments.

If you have any questions, do not hesitate to ask. We are here to help and be your guide on this journey.

\begin{quote}
The syllabus includes key information about the course schedule, assignments, and policies. Please read the full course syllabus, which you will find in Moodle. For information on how to navigate through this course on Moodle, see \href{https://trinitywestern.teamdynamix.com/TDClient/1904/Portal/KB/?CategoryID=8214}{here}.
\end{quote}

\hypertarget{course-description}{%
\subsection*{Course Description}\label{course-description}}
\addcontentsline{toc}{subsection}{Course Description}

Introduces theories and competencies related to learning and thriving in a digital world. Explores how learners are situated in `the digital' throughout their lives and how they can use digital technologies to enhance and enrich their experience of learning, working, and playing. Learners will begin to build a curated digital footprint, initiate and develop personal and professional learning networks; develop competencies to allow them to evaluate and choose digital platforms and tools that are safe and ethical; and explore how to use digital technologies to discover, curate, connect, and share knowledge with their communities.

\hypertarget{meet-your-instructors}{%
\subsection*{Meet Your Instructors}\label{meet-your-instructors}}
\addcontentsline{toc}{subsection}{Meet Your Instructors}

{[}insert{]}

\hypertarget{course-notes}{%
\section*{Course Notes}\label{course-notes}}
\addcontentsline{toc}{section}{Course Notes}

\hypertarget{how-to-navigate-this-book}{%
\subsection*{How To Navigate This Book}\label{how-to-navigate-this-book}}
\addcontentsline{toc}{subsection}{How To Navigate This Book}

Take a moment to experiment with the controls in the toolbar at the top of the page. You can search this book for a word or phrase (for example, to look up a definition). To move quickly to different portions of the book, click on the appropriate chapter or section in the table of contents on the left. The buttons at the top of the page allow you to show/hide the table of contents, search the book, adjust the typeface, the font size, and the background colour to make the text easier to read.

\includegraphics{assets/course-intro/menu.png}

The faint left and right arrows at the sides of each page (or bottom of the page if it's narrow enough) allow you to step to the next/previous section. Here's what they look like:

\includegraphics{assets/course-intro/left_arrow.png} \includegraphics{assets/course-intro/right_arrow.png}

You can also download an offline copy of this books in a pdf format. If you are having any accessibility or navigation issues with this book, please reach out to your instructor or our online team at \href{mailto:elearning@twu.ca}{\nolinkurl{elearning@twu.ca}}

\hypertarget{course-units}{%
\subsection*{Course Units}\label{course-units}}
\addcontentsline{toc}{subsection}{Course Units}

This course is organized into 10 units. Each unit of the course will provide you with the following information:

\begin{itemize}
\tightlist
\item
  A general overview of the key concepts that will be addressed during the unit.
\item
  Specific learning outcomes and topics for the unit.
\item
  Learning activities to help you engage with the concepts. These often include key readings, videos, and reflective prompts.
\item
  The Assessment section provides details on assignments you will need to complete throughout the course to demonstrate your understanding of the course learning outcomes.
\end{itemize}

\begin{caution}
 Note that assessments, including assignments and discussion posts will
 be submitted in Moodle. See the Assessment tab in Moodle for assignment
 details and dropboxes.
 \end{caution}

\hypertarget{course-activities}{%
\subsection*{Course Activities}\label{course-activities}}
\addcontentsline{toc}{subsection}{Course Activities}

Below is some key information on features you will see throughout the course.

\begin{reflect}
\textbf{\emph{Learning Activity}}\\
This box will prompt you to engage in course concepts, often by viewing
resources and reflecting on your experience and/or learning. Most
learning activities are ungraded and are designed to help prepare you
for the assessment in this course.
\end{reflect}

\begin{assessment}
\textbf{\emph{Assessment}}\\
This box will signify an assignment you will submit in Moodle. Note that
assignments demonstrate your understanding of the course learning
outcomes. Be sure to review the grading rubrics for each assignment.
\end{assessment}

\begin{progress}
\textbf{\emph{Checking Your Learning}}\\
This box is for checking your understanding, to make sure you are ready
for what follows.
\end{progress}

\begin{feedback}
\textbf{\emph{Note}}\\
This box signifies key notes, important quotes, or case students. It may
also warn you of possible problems or pitfalls you may encounter!
\end{feedback}

\hypertarget{introduction-to-digital-literacies-for-online-learning}{%
\chapter{Introduction to Digital Literacies for Online Learning}\label{introduction-to-digital-literacies-for-online-learning}}

\hypertarget{overview}{%
\section*{Overview}\label{overview}}
\addcontentsline{toc}{section}{Overview}

Welcome to Unit 1 of Learning with Technology! This course will introduce you to some ideas related to living, learning, and working in our digitally-saturated society. It is our intent to provide you with opportunities to start your university career with an emerging set of skills and literacies related to digital tools for learning. Within your academic pursuits, you will encounter a vast amount of information, and integrating digital tools into your learning journey might be difficult. Your chosen discipline will provide ample learning possibilities, and incorporating digital tools to enhance your learning may prove challenging. This course will give you a head start on using digital tools to build a workflow that will allow you to stay organized and to make your process of learning visible for yourself and your instructors. We will also lead you through readings and thoughts about your digital identity, privacy and security, and sharing your new knowledge in ethical ways.

There will be two primary branches of the course and the tools that we will show you. The first branch will be a workflow that is private to you because it takes place primarily on your own computer, and the second branch is shared as publicly as you are comfortable sharing. You will have control over how public your work is, but we will think about the importance of sharing knowledge and how to do that easily and in ways that preserve your `ownership' over your work.

During this first week, there will be both theoretical and practical work for you to do. In order to build a theoretical understanding of digital tools for learning, we will explore the idea of \emph{the digital} in the context of contemporary society. At the same time, there are some important practicalites to manage in order to get set up for the course, so we will lead you through installing some apps on your computer that you will use extensively in this course, and which hopefully will become the backbone of your digital workflow throughout your time in higher education and beyond.

\hypertarget{topics}{%
\subsection*{Topics}\label{topics}}
\addcontentsline{toc}{subsection}{Topics}

This unit is divided into the following topics:

\begin{enumerate}
\def\labelenumi{\arabic{enumi}.}
\tightlist
\item
  Understanding the Digital
\item
  Digital Literacies
\item
  Digital Safety
\item
  Starting your Workflow
\end{enumerate}

\hypertarget{learning-outcomes}{%
\subsection*{Learning Outcomes}\label{learning-outcomes}}
\addcontentsline{toc}{subsection}{Learning Outcomes}

When you have completed this unit, you should be able to:

\begin{itemize}
\tightlist
\item
  Describe your engagement with digital technology
\item
  Apply digital tools to support learning in an academic environment
\item
  Explain what digital literacies mean for you in an academic and professional context
\item
  Examine your digital footprint
\item
  Build your professional online biography
\item
  Examine privacy concerns related to various platforms and tools
\item
  Describe how to protect yourself and others in the digital environment.
\end{itemize}

\hypertarget{activity-checklist}{%
\subsection*{Activity Checklist}\label{activity-checklist}}
\addcontentsline{toc}{subsection}{Activity Checklist}

Here is a checklist of learning activities you will benefit from in completing this unit. You may find it useful for planning your work.

\begin{reflect}
During this unit you will:

\begin{itemize}
\tightlist
\item
  create a Visitors and Residents diagram
\item
  download and install \href{https://obsidian.md}{Obsidian}
\item
  download and open the course vault in Obsidian

  \begin{itemize}
  \tightlist
  \item
    activate the plugins that came with the Obsidian vault
  \end{itemize}
\end{itemize}

You will be directed to complete these activities as they come.
\end{reflect}

\begin{assessment}
{Assessment}

\begin{itemize}
\tightlist
\item
  See the Assessment section in Moodle for assignment details and due dates.
\end{itemize}
\end{assessment}

\hypertarget{resources}{%
\subsection*{Resources}\label{resources}}
\addcontentsline{toc}{subsection}{Resources}

\begin{itemize}
\tightlist
\item
  All resources will be provided online in the unit.
\end{itemize}

\hypertarget{understanding-the-digital}{%
\section{Understanding the Digital}\label{understanding-the-digital}}

We begin Unit 1 with an introduction to the idea of \emph{the digital}. You may recognize that digital tools are deeply embedded in modern society. It is not uncommon for people of all ages to interact with apps and tools that claim to connect people in conversations or networks, or to perform complex tasks for work, or to control various systems in our vehicles. Digital technology is really everywhere we look. Thinking about these tools is one way to conceptualize how we interact with digital tools, but we can also recognize that our social practices and norms have been impacted by digital tools. An example of this, at least in North America, is that the names of companies have become verbs. If people want to learn something about a topic, they \emph{Google} it. It is deeper than that, though, as the COVID-19 pandemic led to many jurisdictions relying on mobile phones for allowing people to confirm their vaccination status in order to access restaurants or other public venues in the midst of restrictions creating barriers for those who don't have mobile phones. Mobile phones are often essential tools for communication, social media, internet browsing, messaging, entertainment, photography, navigation, online shopping, mobile banking, productivity, and health and fitness management. In other cases, such as in social media, it is almost impossible to participate in public discourse without access to technology.

Modern universities are also deeply impacted by \emph{the digital}. Every system involved in higher education has been digitized in some manner, including recruitment, accounting, and fundraising. As you begin your university career, here are some digital systems you will likely encounter:

\begin{itemize}
\tightlist
\item
  courses are designed and often delivered digitally,
\item
  course logistics (discussion forums, assignment submissions, quizzes, gradebooks) happen in large digital tools called learning management systems (LMS) or virtual learning environments (VLE) (e.g.~Moodle),
\item
  assignments must often be created digitally (word processors, presentation software, video editors, website builders),
\item
  research data is gathered, stored, analyzed, and shared digitally
\end{itemize}

There are many other processes and procedures that rely on \emph{the digital} in higher education, but the important thing for you to realize as you begin your higher education journey is that there are many tools that you will be required to learn and use throughout your journey. Some are more obvious, like word processors, presentation software, email, the library website, and LMSs, but some are less obvious and won't necessarily be taught specifically, other than in this course.

Some of the digital tools we will introduce to you will help you build a \emph{workflow} for you to manage the huge amount of information and resources that you will have to sort through to complete many of your assignments. You will learn to use AI to find \emph{relevant} resources on whatever your topic might be. As you know from searching Google, a simple search of the web can turn up thousands or millions of hits, but there are tools that can help you highlight the 20 most relevant resources in just a few clicks. Once you find resources, we will show you tools that will allow you to track all your references, create citations in your writing quickly and easily, and then create a perfectly formatted reference list. Do not waste your time creating your own bibliographies! This one tool will save you days and likely weeks of work during your degree (quite literally). We will show you another tool that will allow you to make connections between ideas and notes so that you build a network of connected ideas. Curating this network of ideas is possibly one of the most useful things you can do in higher ed.~You will end up with a searchable network of everything you've learned, and be able to visualize it at the click of a button. We will help you think through the implications of how you present yourself on the web so that you can make wise decisions about what you share and how you share it. We will also help you make connections on the web that could become a key resource for your learning and working in your career.

\hypertarget{activity}{%
\subsection*{Activity:}\label{activity}}
\addcontentsline{toc}{subsection}{Activity:}

\begin{reflect}
Head over to \href{https://twu.discourse.group}{the Learning Hub}, which is an app called \emph{Discourse} that we use to build community among learners who do not attend one of the TWU campuses in Langley or Richmond, BC. Find the \emph{Leadership/Media and Communications 101} category and respond to the \emph{Welcome} forum.
\end{reflect}

\hypertarget{digital-literacies}{%
\section{Digital Literacies}\label{digital-literacies}}

\begin{quote}
\begin{definition}[Digital Literacy]
\protect\hypertarget{def:diglit}{}\label{def:diglit}Digital literacy is a person's knowledge, skills, and abilities for using digital tools ethically, effectively, and within a variety of contexts in order to access, interpret, and evaluate information, as well as to create, construct new knowledge, and communicate with others. \citep{digitallearningadvisorycommitteePostSecondaryDigitalLearning2022}
\end{definition}
\end{quote}

Literacy, as we commonly understand it, is the ability to \emph{understand} the meaning of texts. It is more than just being able to `read'. In the same way, digital literacy is the ability to make meaning using digital tools. It is more than simply being able to post to Instagram or TikTok, or whatever app you might use. As the definition above indicates, digital literacy involves using tools \emph{ethically}, to \emph{access, interpret, evaluate, create, construct, and communicate} information and knowledge.

One way to start thinking about digital literacy is to create a map of the apps and tools that you use, how you use them, and what traces of your presence you leave behind on the web. We call this a \emph{Visitors and Residents Diagram}. To complete this activity, you will need to do a little bit of setup, as follows.

\hypertarget{activity-1}{%
\subsection*{Activity}\label{activity-1}}
\addcontentsline{toc}{subsection}{Activity}

\begin{reflect}
{Install Obsidian}

Obsidian is a free and open source note-taking and mind-mapping app.

\begin{enumerate}
\def\labelenumi{\arabic{enumi}.}
\tightlist
\item
  Go to \href{https://obsidian.md/download}{obsidian.md} and \href{https://help.obsidian.md/Getting+started/Download+and+install+Obsidian}{follow these instructions to install Obsidian on your computer.}

  \begin{itemize}
  \tightlist
  \item
    It is recommended that you use a computer, rather than a mobile phone to install Obsidian, but please let your instructor or facilitator know if you are on mobile.
  \item
    You do NOT need to purchase any upgrades such as \textbf{Obsidian Sync} or \textbf{Obsidian Publish}.
  \end{itemize}
\item
  Work through the \textbf{Getting Started} section of the Obsidian help pages starting with \textbf{\href{https://help.obsidian.md/Getting+started/Create+a+vault}{Create a vault}}.

  \begin{itemize}
  \tightlist
  \item
    When you create the vault in this step, we recommend that you name it \textbf{TWU} or something similar. Later, you can create as many vaults as you would like.
  \end{itemize}
\item
  It is recommended that you \href{https://help.obsidian.md/Getting+started/Sync+your+notes+across+devices}{use one of the sync services listed here} so that your files are backed up.
\end{enumerate}
\end{reflect}

Obsidian will become a backbone of this course as we will use it to learn how the web works and give you a workflow that will help you stay organized. One of the advantages of Obsidian is that everything you do in the app happens on your own computer, rather than \emph{the cloud}, which is just another way of saying \emph{someone else's computer}. However, the drawback to that is that you need to ensure that you have a backup of your vaults in a secure location, either one of the sync services mentioned in step 3, above, or another backup system. \href{https://twu.discourse.group/c/ldrs101/10}{Please check the Learning Hub} or talk your your instructor or facilitator for help with this.

To give you a head start, we have created a \textbf{starter vault} for you to download and use. Follow the instructions below to access the starter vault.

\begin{reflect}
{Download the Starter Vault}

\begin{enumerate}
\def\labelenumi{\arabic{enumi}.}
\tightlist
\item
  \href{https://github.com/twu-innovation/ldrs101-vault/archive/refs/heads/main.zip}{Click this link to download the vault.}

  \begin{itemize}
  \tightlist
  \item
    This will download a file called \texttt{ldrs101-vault-main.zip} to your computer.
  \end{itemize}
\item
  Move the file to your \texttt{Documents} folder.
\item
  Unzip or extract the contents of the file.
\item
  Rename the folder to \texttt{ldrs101}.
\item
  Open the \textbf{Obsidian} app and click the \texttt{Open\ another\ vault} icon in the bottom left corner.
\end{enumerate}

\includegraphics{assets/digital-literacy/obsidian2.png}

\begin{enumerate}
\def\labelenumi{\arabic{enumi}.}
\setcounter{enumi}{5}
\tightlist
\item
  Choose the \texttt{ldrs101} folder, and then Obsidian will ask you to trust this vault. Click `Trust author and enable plugins'.
\end{enumerate}
\end{reflect}

\hypertarget{digital-literacies-1}{%
\section{Digital Literacies}\label{digital-literacies-1}}

In this unit we've started discussing `the digital', you have started a blog (part of your PLN), and you've examined your own digital identity. In our final topic, we want to begin to define digital literacies, and continue to build digital skills.

So, what is digital literacy? In the next activity, you will start to unpack this term and prepare your own initial definition of digital literacy.

\hypertarget{activity-definining-digital-literacy}{%
\subsection*{Activity: Definining Digital Literacy}\label{activity-definining-digital-literacy}}
\addcontentsline{toc}{subsection}{Activity: Definining Digital Literacy}

\begin{reflect}
Purpose: Search for definitions of digital literacy and digital skills on the web to identify the difference.

\begin{enumerate}
\def\labelenumi{\arabic{enumi}.}
\tightlist
\item
  Read Wikipedia's definition of \href{https://en.wikipedia.org/wiki/Digital_literacy}{Digital literacy} -- Is this a good description?
\item
  Scan the \href{https://twitter.com/i/flow/login?redirect_after_login=\%2Fhashtag\%2Fdiglit\%3Fs\%3D03}{\#diglit} hashtag on Twitter -- Did you find any valuable links to defining digital literacy?
\item
  Conduct a Google search for ``digital literacy.'' Select a few definitions you like and record the urls, for example by adding these to your browser bookmarks.
\item
  Conduct a Google search for ``digital skills.'' Select one or two definitions you like and record the urls.
\item
  Conduct a Google search for ``digital fluency'' Select one or two definitions.
\item
  What are the differences between digital literacies, digital fluency, and digital skills? How are these concepts related?
\item
  Read: \href{http://pomo.com.au/blog/digital-literacy/}{What is digital literacy?} published by POMO -- Is this a reliable source?
\item
  How would you rate the academic quality of the definitions you found (e.g.~low / high quality)?
\item
  What did you discover? Share your thoughts and experiences by posting on the LDRS101 Discourse channel. For example:
\end{enumerate}

\begin{itemize}
\tightlist
\item
  The major difference between digital skills and literacies is \ldots{}
\item
  I didn't realise that \ldots{}
\item
  For me, digital literacy means \ldots''
\end{itemize}

Note: Your comment will be displayed in the course feed.
\end{reflect}

\hypertarget{digital-skills-versus-literacies}{%
\subsection{Digital skills versus literacies}\label{digital-skills-versus-literacies}}

``Saying that any digital tool teaches us digital literacies is like saying a pen or a keyboard teaches us writing.''
---Maha Bali
\citet{Bali_Maha}

``Digital literacies are not solely about technical proficiency but about the issues, norms, and habits of mind surrounding technologies used for a particular purpose.''
---Doug Belshaw, Educational researcher

\hypertarget{activity-digital-skills-versus-literacies}{%
\subsection*{Activity: Digital skills versus literacies}\label{activity-digital-skills-versus-literacies}}
\addcontentsline{toc}{subsection}{Activity: Digital skills versus literacies}

\begin{reflect}
In this activity, we will review an article on the difference between digital skills and digital literacies using Hypothes.is -- an online social annotation technology.

Purpose
Read an online article and annotate it using Hypothes.is.

Tasks

\begin{enumerate}
\def\labelenumi{\arabic{enumi}.}
\tightlist
\item
  Read the Hypothes.is ``Quick start guide for students''
\item
  Create an account on Hypothes.is. Here is the registration link
  We recommend that you use the Chrome browser (download here) and install the Hypothes.is extension. Alternatively, you can annotate web pages directly from the Hypothes.is website by pasting the link into the text area after you have logged into the site. If you are working on a mobile device, please follow these instructions: How to use Hypothes.is on mobile devices.
\item
  Read the following article: Knowing the Difference Between Digital Skills and Digital Literacies, and Teaching Both
\item
  Activate the annotations after logging in to Hypothes.is and click on the search icon () and enter the course code (LiDA101) to filter posts for this course from the public feed.
\item
  Annotate or reply to posts by visiting the annotation page (You will need to be logged into the Hypothes.is site to post.)
\item
  Remember to tag your posts using the course code: LDRS101 (The course tag is required to harvest posts for the course feed.)
\end{enumerate}
\end{reflect}

\hypertarget{activity-researching-a-definition}{%
\subsection*{Activity: Researching a definition}\label{activity-researching-a-definition}}
\addcontentsline{toc}{subsection}{Activity: Researching a definition}

\begin{reflect}
This course is facilitated and enabled by resources that learners can access on the Internet. This involves the ability to search, evaluate, and select resources to support your learning. A key feature of a PLE is the ability to share useful and valuable information with your peers.

In this activity, you will complete your first resource bank activity, where you are invited to search for additional resources to help you refine and develop your own definition of digital literacies, and to reflect on how this differs from definitions for digital skills. The resource bank activity builds on your initial search on the previous page by inviting you to share links to valuable resources you source on the Internet. The resource bank provides the technology to share these resources with our LDRS 101 learning community.

The resource bank for this course is hosted on \ldots(Obsidian? Evernote? Moodle wiki? Google doc?) is an open source social bookmarking application which enables users to share and annotate links to resources they find on the web. It uses a tag system to group resources according to different topics. The tag system helps users to locate links to resources on the site using the same tag.

The purpose of this activity is to find and select online definitions for digital literacies and digital skills, and to record and share these links using the OERu resource bank. This activity is designed to give you the opportunity to familiarise yourself with our shared resource bank, which is hosted on the bookmarks.oeru.org site. Note the format of the resource bank activity; in particular, scroll down to the footnote area at the bottom of this activity to see the required and recommended tags.

Tasks

\textbf{Search}
1. Conduct a general Google search to find a few definitions for digital literacies and digital skills (narrow your search to three or four resources you find valuable for both concepts.)
\textbf{Select}
1. Select the best resource link for each concept -- think about why you chose these links.
\textbf{Create an account on bookmarks}
1. Click on the register link on bookmarks.oeru.org
1. The code you need for the anti-spam question is contained in the instructions for Session 1.
1. For additional help in using bookmarks, consult the support site.
\textbf{Record, annotate and tag your bookmarks}
1. Log in to bookmarks.oeru.org and click on the ``Add'' link to record your best resource link describing digital literacies, and your best resource link for describing digital skills.
1. Remember to add the required tags and suggested tags (see below in the footnote area).
- Notes about tags: Tags should be lowercase without spaces. When adding more than one tag, these should be separated using commas. The lida101a2-1 refers to the first activity of the second learning pathway of the lida101 course.
- Important: Remember to make your bookmark public (this is located next to the privacy section in the edit window when adding a bookmark).
1. Include a short description summarising the link, including why you recommend the resource.
\textbf{Explore the bookmark tags for digital literacy and digital skills}
1. Click on the ``Tags'' link located at the top of the page at bookmarks.oeru.org.
1. Search for resources tagged: digital literacy, or with the activity number: lida101a2-1 (Note, remember to select ``all bookmarks'' so that your search is not restricted to your own bookmarks.)
1. Explore a few bookmarks which have been submitted by fellow learners. If you find a resource you like, vote for the resource by clicking on the ``thumbs-up'' graphic. Click on the ``copy'' link to add useful bookmarks to your personal bookmark account on bookmarks.oeru.org
1. Search for resources tagged: digital skills (Remember to select ``all bookmarks'').
1. Explore a few bookmarks which have been submitted by fellow learners, and vote for those that you find useful.
\end{reflect}

\hypertarget{activity-forum-why-digital-literacy-matters}{%
\subsection*{Activity: Forum -- Why digital literacy matters}\label{activity-forum-why-digital-literacy-matters}}
\addcontentsline{toc}{subsection}{Activity: Forum -- Why digital literacy matters}

\begin{reflect}
A key component of digital literacy and networked learning relates to the ability to engage meaningfully in online learning communities.

This learning activity will provide you with the opportunity to familiarise yourself with the Moodle discussion boards that are common in TWU online learning discussions.

Purpose of discussion activity:
To reflect on the reasons why digital literacy matters to you, and to demonstrate your digital skills using a discussion forum.

TWU online courses often use Moodle discussion forums as the main discussion platform for learners to engage in course conversations asyncrhonously.

\begin{itemize}
\tightlist
\item
  info re oeru forums - replace with Discourse?
  This is a community managed platform using the open source Discourse software. The system awards badges for a range of activities. Authentic activity increases the user's trust levels which will enable more software features for participants as your trust levels increase over time. Therefore it's a good time to get started in earning your first badges and improving your trust levels early in the course.
\end{itemize}

Watch the following video and jot down the reasons why digital literacy matters to you, then complete the steps which follow.
\url{https://www.youtube.com/watch?v=p2k3C-iB88w\&t=3s}
update video?

Steps

\begin{enumerate}
\def\labelenumi{\arabic{enumi}.}
\tightlist
\item
  Log in to forums.oeru.org (Sign up to register a new account if your don't already have one.)
\item
  Complete your profile page (Click on your user icon on the top right-hand side of your screen and select the settings cog ().)
\item
  Spend a few minutes to explore the badges you can earn on the discourse platform. (Click on the badge for details.) By the end of this learning pathway you should have attained the following badges:
\end{enumerate}

\begin{itemize}
\tightlist
\item
  Autobiographer
\item
  First-like
\item
  First mention
\item
  First quote
\end{itemize}

\begin{enumerate}
\def\labelenumi{\arabic{enumi}.}
\tightlist
\item
  Post a contribution to the discussion on digital literacies and why they are important for you
\item
  Remember to post one or two replies to interesting contributions (You should also ``like'' good contributions, use \citet{username} when replying, and if appropriate quote a reply when responding.)
\end{enumerate}
\end{reflect}

\hypertarget{assessment-and-scope-of-digital-literacies}{%
\subsection{Assessment and scope of digital literacies}\label{assessment-and-scope-of-digital-literacies}}

Digital literacies for academic learning involves more than Facebook, Snapchat or Twitter and the associated technical skills in using these technologies.

As you explore the concept, you will find online resources which confuse digital skills with digital literacies. The activities which follow aim to provide an initial introduction to the wide range of digital literacies associated with academic learning. We will explore the concept of digital literacies in greater depth as we progress with the course. When exploring these online resources, we encourage you to differentiate between skills and literacies and to develop a critical disposition. Digtial literacies involve issues, norms, and habits of mind surrounding technologies used for a particular purpose. However, these literacies are closely related to technical proficiency in using a range of digital applications

\hypertarget{activity-am-i-digitally-literate}{%
\subsection*{Activity: Am I Digitally Literate?}\label{activity-am-i-digitally-literate}}
\addcontentsline{toc}{subsection}{Activity: Am I Digitally Literate?}

\begin{reflect}
Digital literacies encompasses a wide range of capabilities which extend beyond the digital skills associated with different technologies.

\begin{enumerate}
\def\labelenumi{\arabic{enumi}.}
\item
  Study the graphic depicting the Seven elements of digital literacies (scroll down the page to view the graphic). (replace)
\item
  Jot down one or more technologies or tools you would recommend for each of the seven elements in the graphic and assess your competence in using each particular technology/tool (eg below average, average, above average and excellent).
\item
  Next, use your searching skills to discover the range and scope of digital literacies (Don't spend more than 15 to 20 minutes on the self-assessment activity).
\end{enumerate}

\begin{itemize}
\tightlist
\item
  Conduct a Google search using: ``digital literacy self-assessment''
\item
  Choose a link to conduct a self-assessment of your digital literacy.
  Alternatively, you can:
\item
  Try the iTest developed by the University of Exeter or
\item
  Explore the What is digital literacy? page of the Digital Literacies Toolkit developed by the University of Southampton.
\end{itemize}

Questions to consider

\begin{itemize}
\tightlist
\item
  Did the self-assessment you chose focus on digital skills or digital literacies?
\item
  What did you learn from this exercise?
\item
  Share your thoughts by posting on Discourse:
\end{itemize}

Note: Your comment will be displayed in the course feed.
\end{reflect}

\hypertarget{activity-pln-mapping-challenge}{%
\subsection*{Activity: PLN mapping challenge}\label{activity-pln-mapping-challenge}}
\addcontentsline{toc}{subsection}{Activity: PLN mapping challenge}

\begin{reflect}
In this activity you will publish a blog post including your personal definition of digital literacies and complete the digital visitor / digital resident personal learning network (PLN) mapping exercise.

\begin{enumerate}
\def\labelenumi{\arabic{enumi}.}
\item
  Read the Digital Visitor and Resident article on Wikipedia
\item
  Watch David White's video on visitors and resident mapping
\item
  Explore the visitors and residents map below (click on the image for a larger view).
\end{enumerate}

A Visitors \& Residents map of online engagement.jpg

Review one or two examples of the following digital visitor and resident maps created by learners. (Click on the image for a larger view and image attributions. Remember that your ``back button'' will take you back to this page or you can open the image links a new tab or window.)

insert images from \url{https://course.oeru.org/lida101/learning-pathways/introduction-to-digital-literacies/tasks-pln-mapping-challenge/}

\begin{enumerate}
\def\labelenumi{\arabic{enumi}.}
\tightlist
\item
  Create your own Personal Learning Network (PLN) map. You can generate your PLN map using your preferred graphics software, slideshow software, or draw your map free hand and then take a photograph for posting on your blog. Institutional quadrants in this context refers to your professional environment and/or engagement in formal learning contexts. (Note that the final assessment requires your PLN map to be generated using digital tools, but for the purposes of this post you can use a hand drawn graphic as a draft before finalising this post for assessment.)
\item
  Revisit your original definition of digital literacy and think about how you would like to refine and improve your first attempt.
\item
  Prepare a revised definition of digital literacy and what digital literacy means for you.
\item
  Complete today's LiDA photo challenge and share your reflection and image on mastodon.oeru.org or Twitter. Remember to include the following two hashtags in the text of your tweet: \#LiDA101 and \#lida101photo
\item
  Consider the digital literacies you would like personally to improve on this course based on the seven elements of digital literacies
\end{enumerate}
\end{reflect}

\hypertarget{activity-blog-digital-literacies-for-online-learning}{%
\subsection*{Activity: Blog: Digital Literacies for Online Learning}\label{activity-blog-digital-literacies-for-online-learning}}
\addcontentsline{toc}{subsection}{Activity: Blog: Digital Literacies for Online Learning}

\begin{reflect}
Publish a blog post responding to four requirements:

Your personal definition of digital literacies justified from your reading of the literature (about 100 to 150 words)
Describe what digital literacies mean for you in a sentence.
Upload an image of your PLN map in your blog post.
Summarise an action plan for improving your digital literacies. Identify the literacies you plan to improve including the reasons why and how you aim to achieve this.
Ensure that your references are cited appropriately.
Scan the course feed for blog posts and post a comment on two posts from your peers.
Notes

Remember to tag or label your post using the course code: LiDA101. (If you omit this step, we will not be able to harvest your post for the course feed.)
\end{reflect}

\hypertarget{activity-researching-a-definition-1}{%
\subsection*{Activity: Researching a definition}\label{activity-researching-a-definition-1}}
\addcontentsline{toc}{subsection}{Activity: Researching a definition}

\begin{reflect}
\end{reflect}

\hypertarget{summary}{%
\section*{Summary}\label{summary}}
\addcontentsline{toc}{section}{Summary}

In this first unit, you have had the opportunity to learn about \ldots{}

\hypertarget{assessment}{%
\section*{Assessment}\label{assessment}}
\addcontentsline{toc}{section}{Assessment}

\begin{assessment}
{Quizzes 1 \& 2}

After completing this unit, including the learning activities, you are asked to complete \ldots{}
\end{assessment}

\hypertarget{checking-your-learning}{%
\section*{Checking your Learning}\label{checking-your-learning}}
\addcontentsline{toc}{section}{Checking your Learning}

\begin{progress}
Before you move on to the next unit, check that you are able to:

\begin{itemize}
\tightlist
\item
  Describe your engagement with digital technology
\item
  Apply digital tools to support learning in an academic environment
\item
  Explain what digital literacies mean for you in a tertiary education context
\item
  Examine your digital footprint
\item
  Build your professional online biography
\item
  Examine privacy concerns related to various platforms and tools
\item
  Describe how to protect yourself, other students and colleagues, to stay safe in the digital environment.
\end{itemize}
\end{progress}

This is text Word To define this is more text

\hypertarget{discovering-and-curating-resources}{%
\chapter{Discovering and Curating Resources}\label{discovering-and-curating-resources}}

\hypertarget{overview-1}{%
\section*{Overview}\label{overview-1}}
\addcontentsline{toc}{section}{Overview}

\ldots{}

See Definition \ref{def:diglit}

\hypertarget{topics-1}{%
\subsection*{Topics}\label{topics-1}}
\addcontentsline{toc}{subsection}{Topics}

This unit is divided into the following topics:

\begin{enumerate}
\def\labelenumi{\arabic{enumi}.}
\tightlist
\item
  Finding \& Selecting Resources
\item
  Citation Management
\end{enumerate}

\hypertarget{learning-outcomes-1}{%
\subsection*{Learning Outcomes}\label{learning-outcomes-1}}
\addcontentsline{toc}{subsection}{Learning Outcomes}

When you have completed this unit, you should be able to:

\begin{itemize}
\tightlist
\item
  \ldots{}
\item
  \ldots{}
\end{itemize}

\hypertarget{activity-checklist-1}{%
\subsection*{Activity Checklist}\label{activity-checklist-1}}
\addcontentsline{toc}{subsection}{Activity Checklist}

Here is a checklist of learning activities you will benefit from in completing this unit. You may find it useful for planning your work.

\begin{reflect}
{Learning Activities}

\begin{itemize}
\tightlist
\item
  \ldots{}
\item
  \ldots{}
\end{itemize}

\textbf{Note:} The learning activities in this course are designed to prepare you for the graded assigments in this course.\\
You are strongly encouraged to complete them.
\end{reflect}

\begin{assessment}
{Assessment}

\begin{itemize}
\tightlist
\item
  See the Assessment section in Moodle for assignment details and due dates.
\end{itemize}
\end{assessment}

\hypertarget{resources-1}{%
\subsection*{Resources}\label{resources-1}}
\addcontentsline{toc}{subsection}{Resources}

\begin{itemize}
\tightlist
\item
  All resources will be provided online in the unit.
\end{itemize}

\hypertarget{topic}{%
\section{Topic}\label{topic}}

content

\hypertarget{activity-2}{%
\subsection*{Activity:}\label{activity-2}}
\addcontentsline{toc}{subsection}{Activity:}

\begin{reflect}
Watch/read\ldots{}

{Questions to Consider}

After completing the activities above, answer the following questions:

\begin{itemize}
\tightlist
\item
  \ldots{}
\end{itemize}
\end{reflect}

\hypertarget{topic-1}{%
\section{Topic}\label{topic-1}}

content

\hypertarget{activity-3}{%
\subsection*{Activity:}\label{activity-3}}
\addcontentsline{toc}{subsection}{Activity:}

\begin{reflect}
Watch/read\ldots{}

{Questions to Consider}

After completing the activities above, answer the following questions:

\begin{itemize}
\tightlist
\item
  \ldots{}
\end{itemize}
\end{reflect}

\hypertarget{topic-2}{%
\section{Topic}\label{topic-2}}

content

\hypertarget{activity-4}{%
\subsection*{Activity:}\label{activity-4}}
\addcontentsline{toc}{subsection}{Activity:}

\begin{reflect}
Watch/read\ldots{}

{Questions to Consider}

After completing the activities above, answer the following questions:

\begin{itemize}
\tightlist
\item
  \ldots{}
\end{itemize}
\end{reflect}

\hypertarget{summary-1}{%
\section*{Summary}\label{summary-1}}
\addcontentsline{toc}{section}{Summary}

In this unit, you have had the opportunity to learn about \ldots{}

\hypertarget{assessment-1}{%
\section*{Assessment}\label{assessment-1}}
\addcontentsline{toc}{section}{Assessment}

\begin{assessment}
{Quizzes 1 \& 2}

After completing this unit, including the learning activities, you are asked to complete \ldots{}
\end{assessment}

\hypertarget{checking-your-learning-1}{%
\section*{Checking your Learning}\label{checking-your-learning-1}}
\addcontentsline{toc}{section}{Checking your Learning}

\begin{progress}
Before you move on to the next unit, check that you are able to:

\begin{itemize}
\tightlist
\item
  \ldots{}
\item
  \ldots{}
\end{itemize}
\end{progress}

\hypertarget{connecting-ideas-for-learning}{%
\chapter{Connecting Ideas for Learning}\label{connecting-ideas-for-learning}}

\hypertarget{overview-2}{%
\section*{Overview}\label{overview-2}}
\addcontentsline{toc}{section}{Overview}

\ldots{}

\hypertarget{topics-2}{%
\subsection*{Topics}\label{topics-2}}
\addcontentsline{toc}{subsection}{Topics}

This unit is divided into the following topics:

\begin{enumerate}
\def\labelenumi{\arabic{enumi}.}
\tightlist
\item
  Sense-making through Hyperlinks
\item
  Sense-making through Taxonomies
\item
  Note-Taking
\item
  Concept Maps
\item
  Digital Tools to Support Learning
\end{enumerate}

\hypertarget{learning-outcomes-2}{%
\subsection*{Learning Outcomes}\label{learning-outcomes-2}}
\addcontentsline{toc}{subsection}{Learning Outcomes}

When you have completed this unit, you should be able to:

\begin{itemize}
\tightlist
\item
  \ldots{}
\item
  \ldots{}
\end{itemize}

\hypertarget{activity-checklist-2}{%
\subsection*{Activity Checklist}\label{activity-checklist-2}}
\addcontentsline{toc}{subsection}{Activity Checklist}

Here is a checklist of learning activities you will benefit from in completing this unit. You may find it useful for planning your work.

\begin{reflect}
{Learning Activities}

\begin{itemize}
\tightlist
\item
  \ldots{}
\item
  \ldots{}
\end{itemize}

\textbf{Note:} The learning activities in this course are designed to prepare you for the graded assigments in this course.\\
You are strongly encouraged to complete them.
\end{reflect}

\begin{assessment}
{Assessment}

\begin{itemize}
\tightlist
\item
  See the Assessment section in Moodle for assignment details and due dates.
\end{itemize}
\end{assessment}

\hypertarget{resources-2}{%
\subsection*{Resources}\label{resources-2}}
\addcontentsline{toc}{subsection}{Resources}

\begin{itemize}
\tightlist
\item
  All resources will be provided online in the unit.
\end{itemize}

\hypertarget{topic-3}{%
\section{Topic}\label{topic-3}}

content

\hypertarget{activity-5}{%
\subsection*{Activity:}\label{activity-5}}
\addcontentsline{toc}{subsection}{Activity:}

\begin{reflect}
Watch/read\ldots{}

{Questions to Consider}

After completing the activities above, answer the following questions:

\begin{itemize}
\tightlist
\item
  \ldots{}
\end{itemize}
\end{reflect}

\hypertarget{topic-4}{%
\section{Topic}\label{topic-4}}

content

\hypertarget{activity-6}{%
\subsection*{Activity:}\label{activity-6}}
\addcontentsline{toc}{subsection}{Activity:}

\begin{reflect}
Watch/read\ldots{}

{Questions to Consider}

After completing the activities above, answer the following questions:

\begin{itemize}
\tightlist
\item
  \ldots{}
\end{itemize}
\end{reflect}

\hypertarget{topic-5}{%
\section{Topic}\label{topic-5}}

content

\hypertarget{activity-7}{%
\subsection*{Activity:}\label{activity-7}}
\addcontentsline{toc}{subsection}{Activity:}

\begin{reflect}
Watch/read\ldots{}

{Questions to Consider}

After completing the activities above, answer the following questions:

\begin{itemize}
\tightlist
\item
  \ldots{}
\end{itemize}
\end{reflect}

\hypertarget{summary-2}{%
\section*{Summary}\label{summary-2}}
\addcontentsline{toc}{section}{Summary}

In this unit, you have had the opportunity to learn about \ldots{}

\hypertarget{assessment-2}{%
\section*{Assessment}\label{assessment-2}}
\addcontentsline{toc}{section}{Assessment}

\begin{assessment}
{Quizzes 1 \& 2}

After completing this unit, including the learning activities, you are asked to complete \ldots{}
\end{assessment}

\hypertarget{checking-your-learning-2}{%
\section*{Checking your Learning}\label{checking-your-learning-2}}
\addcontentsline{toc}{section}{Checking your Learning}

\begin{progress}
Before you move on to the next unit, check that you are able to:

\begin{itemize}
\tightlist
\item
  \ldots{}
\item
  \ldots{}
\end{itemize}
\end{progress}

\hypertarget{building-your-online-presence}{%
\chapter{Building Your Online Presence}\label{building-your-online-presence}}

\hypertarget{overview-3}{%
\section*{Overview}\label{overview-3}}
\addcontentsline{toc}{section}{Overview}

Welcome to Unit 4! In previous units, you've been introduced to the world of digital literacies and how to use various tools for organizing and connecting ideas. You have started to build a workflow to help you learn more effectively, and have applied the critical skill of metacognition to explain your process for learning.

Now, let's delve into the next phase of our learning journey.

In the second half of the course, you will continue to build your digital skills and use critical thinking to record your learning process. Our attention will move from creating a personal collection of ideas to presenting your learning in a more open platform. Note that you wil decide how public you want to be. We'll also consider why sharing knowledge is important and explore user-friendly ways to do so while still maintaining your control over your work and handling privacy matters. During this unit, you'll get the chance to create a blog, discover your digital footprint, and assess various digital tools.

\hypertarget{topics-3}{%
\subsection*{Topics}\label{topics-3}}
\addcontentsline{toc}{subsection}{Topics}

This unit is divided into the following topics:

\begin{enumerate}
\def\labelenumi{\arabic{enumi}.}
\tightlist
\item
  Personal Learning Environments
\item
  Building a Learning Blog
\item
  My Digital Footprint
\item
  Evaluating Digital Tools
\end{enumerate}

\hypertarget{learning-outcomes-3}{%
\subsection*{Learning Outcomes}\label{learning-outcomes-3}}
\addcontentsline{toc}{subsection}{Learning Outcomes}

When you have completed this unit, you should be able to:

\begin{itemize}
\tightlist
\item
  Create a personalized narrative to document and express your learning process
\item
  Examine your digital footprint and develop a positive digital online identity
\item
  Evaluate digital tools, platforms, and interactions based on ethical principles
\item
  Critically evaluate the affordances and restraints of digital tools and platforms
\item
  Identify the digital skills needed in your field of study
\item
  Describe how to protect yourself, other students and colleagues, to stay safe in the digital environment.
\item
  Practice evaluative judgment to document your process of learning in complex domains of knowledge
\end{itemize}

\hypertarget{activity-checklist-3}{%
\subsection*{Activity Checklist}\label{activity-checklist-3}}
\addcontentsline{toc}{subsection}{Activity Checklist}

Here is a checklist of learning activities you will benefit from in completing this unit. You may find it useful for planning your work.

\begin{reflect}
{Learning Activities}

\begin{itemize}
\tightlist
\item
  Create a new blog on WordPress and personalise your blog site.
\item
  Document and share your learning experience by posting a blog entry.
\end{itemize}

\textbf{Note:} The learning activities in this course are designed to prepare you for the graded assigments in this course. You are strongly encouraged to complete them.
\end{reflect}

\begin{assessment}
{Assessment}

\begin{itemize}
\tightlist
\item
  See the Assessment section in Moodle for assignment details and due dates.
\end{itemize}
\end{assessment}

\hypertarget{resources-3}{%
\subsection*{Resources}\label{resources-3}}
\addcontentsline{toc}{subsection}{Resources}

\begin{itemize}
\tightlist
\item
  All resources will be provided online in the unit.
\end{itemize}

\begin{feedback}
\textbf{\emph{Resource Reminders}}\\
- Be sure to add resources to your \emph{Zotero library} you find
relevant to your personal learning journey.\\
- Your peers are a resource! Reach out to your online community for
support, and be sure to share resources and insights.
\end{feedback}

\hypertarget{personal-learning-environments}{%
\section{Personal Learning Environments}\label{personal-learning-environments}}

The primary purpose of this topic is to enable you to set up your course blog, which will form the hub of your Personal Learning Environment (PLE). Blog posts are a useful way of reflecting on your learning and a means to network with your peers. It also provides our TWU learning community with a way to see how you are engaging with the course and to help where we can. The purpose of a PLE is to put the learner at the centre of the online learning environment, which will be enabled by establishing a personal blog for this course.

So what is a Personal Learning Environment?

\begin{quote}
`Personal Learning Environments are systems that help learners take control of and manage their own learning. This includes providing support for learners to set their own learning goals, manage their learning; managing both content and process, communicate with others in the process of learning, and thereby achieve learning goals. A PLE may be composed of one or more sub-systems: As such it may be a desktop application, or composed of one or more web-based services.' \href{https://edutechwiki.unige.ch/en/Personal_learning_environment}{PLE Wiki}
\end{quote}

\begin{quote}
A Personal Learning Environment is ``a structure and process that helps learners organize the influx of information, resources and interactions that they are faced with on a daily basis into a personalized learning space or experience. In a PLE, the learner develops an individualized digital identity through the perceptual cues and cognitive affordances that the personal learning environment provides, such as what information to share and when, who to share it with, and how to effectively merge formal and informal learning experiences (\href{https://naerjournal.ua.es/article/view/v6n1-introduction\#}{Castañeda, Cosgrave, Marín, Cronin, 2016}) cited in \href{https://naerjournal.ua.es/article/view/v6n1-introduction}{Personal Learning Environments: Research-Based Practices, Frameworks and Challenges}.
\end{quote}

What elements of the two definitions resonate with you? How do you organize your daily interations and influx of information? How do you share your learning with others? What are your learning goals?

\hypertarget{activity-what-is-a-ple}{%
\subsection*{Activity: What is a PLE?}\label{activity-what-is-a-ple}}
\addcontentsline{toc}{subsection}{Activity: What is a PLE?}

\begin{reflect}
Before you start building your PLE, read the following article:

\begin{itemize}
\tightlist
\item
  \href{assets/u4/U4_7-things-you-should-know-about-PLNs.pdf}{``7 Things you should know about personal learning environments''}
\end{itemize}

{Questions to Consider}

After reading the article, consider the following:

\begin{itemize}
\tightlist
\item
  How do PLEs promote authentic, student-centred learning?
\item
  What are the benefits of a PLE? How would it benefit you?
\item
  What tools do you currently use as part of your learning environment?
\end{itemize}

Finally, consider the approach taken at TWU as it supports inquiry-rich learning. As you watch the short video below, think about how you could use your PLE to enrich your learning at TWU.

\href{https://www.youtube.com/watch?v=SCa9Nt3X1vU}{Watch: \emph{Inquiry-Rich Learning}}
\end{reflect}

\hypertarget{building-a-learning-blog}{%
\section{Building a Learning Blog}\label{building-a-learning-blog}}

In the next activity, you will gain first hand experience in using blog technology for publishing your own course site. You will ``declare'' yourself online using your PLE (as an alternative to posting an introduction in a closed course forum typically used in a conventional online course). Note that TWU online courses often use Moodle Discussion Forums to facilitate conversations. By using a platform such as Wordpress, you can retain the contents of your posts, as well as the comments of your peers. In an LMS (Learning Management System) such as Moodle, you may lose access to what you have posted in discussions, and more importantly, conversations with your peers. As you create your personal blog in WordPress (or your own selective blog site), you control your data and who can see it.

You will retain control of your data and learning outputs generated during this online course, even after the course is completed.
You get to choose:

\begin{itemize}
\tightlist
\item
  The blog service you would like to use, although \textbf{we recommend WordPress as it is supported by TWU}.
\item
  Whether to accept comments on your blog from your peers
\item
  Whether to register your blog for the aggregated course feed so that any posts tagged with the course code (LDRS101) will be harvested for the feed.
\end{itemize}

A key teaching philosophy of this course is to embed the acquisition of new digital literacies into your learning journey. Knowledge of how to use the Internet and social media technologies will better prepare you for life in a digital world. If this is your first time blogging, you should spend time in setting up your personal digital learning environment. Please remember that your course blog and the social media technologies you use on this course are public, and that you take full responsibility for anything you publish. Do not disclose any confidential information and respect the privacy of others. In short, don't say anything that you would not want to read on the Internet.

\hypertarget{activity-setting-up-your-course-blog}{%
\subsection*{Activity: Setting Up Your Course Blog}\label{activity-setting-up-your-course-blog}}
\addcontentsline{toc}{subsection}{Activity: Setting Up Your Course Blog}

\begin{reflect}
As this is a course focusing on digital literacies, you are asked to establish a course blog, as this will improve your skills and enable you to network with your peers. We recommend using WordPress, as it is supported by TWU. WordPress is an open source website builder and is one of the most popular systems out there because of its versatility. If you already have your own website or you have previous experience using WordPress, you may set up your blog on it and skip the set-up steps described below, but you still need to complete the learning activities.

Watch the following video tutorial on how to create a website using WordPress. This video provides you with an overview of all the necessary steps. Do not worry if you start to feel overwhelmed --- we will break these steps down for you one by one in the learning activities in this course.

\begin{quote}
We are here to help you create your site, so do not hesitate to ask for technical support. Below you will find a number of resources, but if you get stuck, please reach out on Discourse, or email \href{mailto:elearning@twu.ca}{\nolinkurl{elearning@twu.ca}}
\end{quote}

To get started on creating your site we suggest the following steps:

\textbf{1. Sign up to create a website}

Go to \textbf{\href{https://create.twu.ca/}{create.twu.ca}} to sign up for your free WordPress site. \emph{Please read all the prompts and instructions carefully!}
Be sure to read the Privacy Statement carefully before clicking ``I Agree.'' The information provided gives you excellent guidance regarding digital citizenship, privacy, and how to build a professional digital persona.

You will be prompted to \textbf{create a domain name}, which is your website's address on the Internet. Often this is referred to as a URL (Uniform Resource Locator). This is what your users will type in their browsers to reach your site. Make sure that you choose a domain name that is related to you, easy to pronounce and spell, and easy to remember. Once you have done that, we suggest you write all this information somewhere you can access it easily -- just in case.

You will also be asked to \textbf{select a theme} for your website. You are free to choose any template you wish. TWU Spark, TWU Hope, and TWU Spartans portfolio are simple to set up and provide easy navigation.

When you choose your theme, your new site will come with a simple menu and instructions for portfolio and website creation.

When you have activated your site (look for a notification in your TWU email), then you are ready to create.

\textbf{2. Explore your dashboard}

The dashboard is the initial area you see when you log in to TWU Create. It's the centre for your site management and where you create content. From the Dashboard you can navigate to content, settings, themes, plugins, and more.

When logged in to TWU Create, you will always have access to an \textbf{admin menu} visible on your sites. From the menu item that is the name of the blog (second from left), you can find the link to the dashboard. While in the dashboard, the same menu can be used to return to the front view of your site.

Determine the difference between the dashboard used for editing and the published view of your blog. (It is important to know the difference because, when you register your blog for the course feed, you must use the url for the public view of your blog).\\
\emph{Progress check:}\\
- Do you know how to open the published (public view) of your blog in a new window?\\
- Have you added a browser bookmark to your dashboard and public version of your blog?

\textbf{Help Tips:} When you are in the site administration area of your site, you can get tips on what you are doing by clicking the ``Help'' menu on the top-right corner. Click on ``Help'' and read through the Overview describing the elements of the dashboard.

insert - Animation showing location of WordPress help tab - from \url{https://create.twu.ca/eportfolios/wordpress/}

\textbf{3. Review your settings}

Review and customise your blog settings from the dashboard according to your preferences.

\begin{itemize}
\tightlist
\item
  \textbf{Enable Categories and Tags}
  We recommend that you enable \textbf{categories} and \textbf{tags} on your blog.\\
  \textbf{Categories} are best used for broad groupings of topics. For example, if you're creating a site that reviews pop culture, you might use categories such as Books, Film, and TV.
\end{itemize}

\textbf{Tags} are more specific keywords that you want to use to associate related content. For example, if you were creating a site that reviews pop culture, you might want to use tags such as science fiction, horror, and action adventure.

You can combine the two! For our review site example, you might be reviewing a romantic comedy. You can assign the broader category Film to the post, then give it some more specific tags such as romantic comedy, or even use the name of the actors and director as tags. People who view that post could use the tags to find related posts around that topic.

\textbf{Set up Comments Settings (Optional)}\\
WordPress comes with a built-in comment system allowing your users to leave comments on your posts. This comment system is great for user engagement, but it can also be targeted by spammers as well. If you don't want comments on your posts, then ensure that the `Allow comments' box is unchecked at the bottom of the editor page.

If you do want comments, but want to manage the spam, you'll need to enable comment moderation on your website.

Visit Settings » Discussions page and scroll down to `Before a comment appears' section. Check the box next to `Comment must be manually approved' option.

\textbf{4. Personalise your blog}

Visit the appearance option on your dashboard and personalise your blog by:\\
- Changing your theme, header image, background colours and/or image\\
- Add at least one widget to your blog. Remember --- ``less is more``. One or two of the following are functional choices: Archives, recent posts, categories or category cloud, and blogs I follow.

You need to hit the \textbf{``Save''} button to save your changes.

\textbf{5. Add a page \& a post}

Pages and posts are where the content is housed on WordPress. The biggest difference between the two is that posts are timestamped, whereas pages are timeless.

\textbf{Pages} are for static content. They do not need a publish date. Use pages you want your visitors to always be able to see that content in that spot, no matter when they visit.

\textbf{Posts} are for timely content. They have a publish date, and they are displayed with the newest content at the top (reverse chronological order) of your site's blog page. Older posts can ``fall off'' the blog page (the content is still kept, but no longer visible). Posts are what you should think of when you hear the term ``blog post.'' Usually posts have a comment section, and this is where viewers can write a comment in response to your post. This may be a handy way to receive feedback from your peers. Also, you can categorize or tag your posts, which is useful to help readers locate posts on your blog.

To add a new Page or Post, click the Pages or Posts menu option and then click the Add New link underneath. Another way is to hover your cursor over Pages or Posts and click the Add new link in the fly-out menu.

\textbf{\emph{BLOG CHALLENGE!}}

\textbf{Edit a Page:} Complete your personal details for display on the ``About'' page of your blog.\\
- \emph{Progress check:} Can you see the updates on your ``about'' page in the published view of your blog?

\textbf{Edit a Blog Post:} Reflect on your experience of this activity on creating a blog.
Click on ``save draft'' (so you can review before publishing live on the web). Your reflection could for example:\\
- Introduce yourself and reflect on what you would like to achieve by maintaining a blog to support your learning\\
- Reflect on what you thought of the activity; Was it easy or hard?\\
- Share links to any additional resources you found useful in completing the tasks.\\
- Provide tips for future learners who will be completing this activity. If you were to set up a new blog again, what would you do differently?\\
- Add anything your readers may find interesting or useful.

\textbf{6. Add media}

Using different types of media to represent your artifacts is a great way to make your portfolio dynamic and keep your audience engaged. Text-heavy pages can get cumbersome regardless of how you arrange it. Media can help with breaking up content or replacing text all together. Consider how you can ``show what you know'' rather than just simply telling. Media can also be an alternative to simply hyperlinking all your artifacts. Instead of sending your audience off to another site or tab, media can be embedded (see \href{https://create.twu.ca/eportfolios/wordpress/media-library/}{Media Library}) to keep your audience contained to your page.

The Media Library on your WordPress site houses the media you upload to your site. WordPress supports a variety of media types such as images, audio, video, and documents. We do suggest that you host your video files in your TWU Microsoft Stream account for optimal playability. Other types of media are typically uploaded and inserted into the text editor when writing a post or page.

\textbf{\emph{BLOG PHOTO CHALLENGE!}}

\textbf{Choose a photo} to add to your blog post.\\
- Be sure you have permission to upload the photo. We suggest using an open licence site, such as \href{https://pixabay.com/}{Pixabay}, \href{https://unsplash.com/}{Unsplash}, \href{https://www.pexels.com/}{Pexels}, \href{https://commons.wikimedia.org/wiki/Main_Page}{Wikimedia Commons}, or \href{https://www.flickr.com/search/?text=pho\&license=2\%2C3\%2C4\%2C5\%2C6\%2C9}{Flickr}.\\
- Alternatively, upload your own photo, take a selfie or ask someone to take a photo of you working on this blog post challenge.

\textbf{\emph{BLOG VIDEO CHALLENGE!}} (Optional)

You may find as you continue the course that you want to share videos you find on the web, or perhaps even your own!

If you're up for the challenge, consider recording a short video introduction and embed this in your blog post.

Here is a tutorial on how to add a Youtube video or embed one of your own videos to your blog:

\textbf{7. Publish!}

Review your draft post and, when you're happy with what you've written, click on the \textbf{``Publish post''} button.

\textbf{7. Share your blog}

Add a \textbf{category} or \textbf{tag} for your post using the course tag: LDRS101

Post in the \textbf{LDRS 101 Discourse forum} to let your peers know the web address of your blog and ask them to post a comment. This will give you the opportunity to experience how comments function on your blog and to test if they are working properly.

\textbf{Additional Customizations}

When you're ready to start customizing your blog and putting content in, check out some tutorials available to you:

\begin{itemize}
\item
  \href{https://www.twu.ca/academics/academic-professional-support/online-learning-resources/wordpress}{TWU's Wordpress Support Page}
\item
  TWU's Wordpress Video Tutorials - new sharepoint address?
\item
  \href{https://wordpress.com/support/}{WordPress Support Website}
\item
  \href{https://www.wpbeginner.com/start-here/}{Beginner's guide for WordPress} by WPBeginner
\item
  \href{https://onlineacademiccommunity.uvic.ca/wordpress-tutorials/}{WordPress Tutorials} from University of Victoria
\end{itemize}

\begin{quote}
If you are confused about anything it is always good to do an initial Google or YouTube search, reach out on Discourse, or email \href{mailto:elearning@twu.ca}{\nolinkurl{elearning@twu.ca}}
\end{quote}
\end{reflect}

Congratulations!! You created your PLE for TWU!

\hypertarget{my-digital-footprint}{%
\section{My Digital Footprint}\label{my-digital-footprint}}

Now that you have started your course blog and introduced yourself online, let's examine your existing online identity. If your future employers were to perform a google search for your name, what would they find? What would you want them to find? As we examine online identities in this topic, we will ask you to consider how you can improve your digital identity in support of your online learning, as well as future employment prospects.

First, let's clarify some key terms.

We need to distinguish between the technical and human elements of online identity. In this course, we are more interested in the human side of online identity, but in part, this is determined by how technology automates the process of building your digital footprint.

\begin{quote}
\textbf{Digital identity} refers to the information utilized by computer systems to represent external entities, including a person, organization, application, or device. When used to describe an individual, it encompasses a person's compiled information and plays a crucial role in automating access to computer-based services, verifying identity online, and enabling computers to mediate relationships between entities. Digital identity for individuals is an aspect of a person's social identity and can also be referred to as online identity. (\href{https://en.wikipedia.org/wiki/Digital_identity}{Wikipedia}: Online).
\end{quote}

\begin{quote}
\textbf{Digital footprint} or digital shadow refers to one's unique set of traceable digital activities, actions, contributions, and communications manifested on the Internet or digital devices. Digital footprints can be classified as either passive or active. The former is composed of a user's web-browsing activity and information stored as cookies. The latter is often released deliberately by a user to share information on websites or social media. While the term usually applies to a person, a digital footprint can also refer to a business, organization or corporation. (\href{https://en.wikipedia.org/wiki/Digital_footprint}{Wikipedia}:Online).
\end{quote}

\hypertarget{activity-what-is-a-digital-footprint}{%
\subsection*{Activity: What is a Digital Footprint?}\label{activity-what-is-a-digital-footprint}}
\addcontentsline{toc}{subsection}{Activity: What is a Digital Footprint?}

\begin{reflect}
Watch the following video and consider the steps you would take to control your digital footprint.
\end{reflect}

\hypertarget{activity-who-am-i-onlineand-why-should-i-care}{%
\subsection*{Activity: Who am I Online\ldots and Why Should I Care?}\label{activity-who-am-i-onlineand-why-should-i-care}}
\addcontentsline{toc}{subsection}{Activity: Who am I Online\ldots and Why Should I Care?}

\begin{reflect}
Read the following articles:

\begin{itemize}
\tightlist
\item
  \href{assets/u4/U4_Understanding-your-Online-Identity-An-Overview-of-Identity.pdf}{Understanding your Online Identity}.
\item
  \href{https://research.com/education/how-to-manage-digital-footprint}{How To Manage Your Digital Footprint: 20 Tips for Students}
\end{itemize}

{Questions to Consider}

Consider the following questions:

\begin{itemize}
\tightlist
\item
  How does your real-world identity differ from your online identity?\\
\item
  What factors inhibit or support the sharing of information in building an online identity?\\
\item
  What is the value of an online identity for learning?
\end{itemize}

\begin{quote}
Reminder: As you view online resources in this course, feel free to annotate and discuss web resources publicly in support of your learning. (Digital Tools: Hypothes.is, Discourse, WordPress, etc.)
\end{quote}

In addition to evaluating who you are online, ask yourself, ``Why Should I Care?''

First, watch the following video, \href{https://www.youtube.com/watch?v=Ro_LlRg8rGg}{Four Reasons to Care About Your Digital Footprint}

Next, select from these resources to inform your views:

\begin{itemize}
\tightlist
\item
  \href{https://www.huffpost.com/entry/students-turn-to-internet_b_3518598}{Students turn to Internet to build online presence, showcase work}, published on Huffingtonpost.
\item
  \href{https://help.open.ac.uk/your-online-presence}{Your Online Presence} published by the Open University
\item
  \href{https://www.careercast.com/career-news/10-ways-build-your-online-identity}{10 Ways to Build your Online Identity}
\item
  UBC's \href{https://digitaltattoo.ubc.ca/}{Digital Tattoo project}
\item
  \href{https://www.internetsociety.org/policybriefs/privacy/}{Policy Brief: Privacy} from the Internet Society
\end{itemize}

Finally, consider how much someone could find out about you from your digital footprints. Here's an interesting video that might cause you to reconsider what you post online.
\end{reflect}

\hypertarget{activity-digital-footprint-audit}{%
\subsection*{Activity: Digital Footprint Audit}\label{activity-digital-footprint-audit}}
\addcontentsline{toc}{subsection}{Activity: Digital Footprint Audit}

\begin{reflect}
In this activity you will audit your own digital footprint in order to find out what exists on the internet about you, and reflect on what you want your online identity to be. Follow the steps below to begin.

\begin{enumerate}
\def\labelenumi{\arabic{enumi}.}
\tightlist
\item
  Conduct a Google search of your own name (using an incognito or private window in Chrome or Firefox). Search for your first name and surname without parenthesis (for example: snow white) and then with parenthesis (for example: ``snow white''). Explore the results of your search.
\item
  Conduct a Google search of your name with the name of current and previous employers.
\item
  Conduct a Google search of your name with the name of previous schools you attended.
\item
  Expand your search to include social media sites, for example: ``snow white'' twitter; ``snow white'' facebook; ``snow white'' youtube etc.
\item
  Note any interesting or surprising findings.
\end{enumerate}
\end{reflect}

\hypertarget{activity-blog-my-digital-footprint}{%
\subsection*{Activity: Blog: My Digital Footprint}\label{activity-blog-my-digital-footprint}}
\addcontentsline{toc}{subsection}{Activity: Blog: My Digital Footprint}

\begin{reflect}
Prepare and publish a short blog post of about 250 to 300 words focusing on what you hope to achieve with your online digital identity for learning. Your post can include:

\begin{enumerate}
\def\labelenumi{\arabic{enumi}.}
\item
  Optional reflection: You may want to include a reflection(s) on the outcomes of your footprint audit. Remember that your blog post is public, so only share what you are comfortable sharing with the world. You don't need to be specific; for example, you can generalise: I am satisfied with my digital footprint because \ldots{} or I would like to improve my digital footprint for learning because \ldots{}
\item
  Professional versus private: Consider how you want to separate your ``private'' online identity from your professional and / or learning identity. If you already maintain an online presence (existing blog or social media accounts) think about how you will separate professional / learning posts from private and social life interactions online. For example, maintaining a separate course or learning blog is one way to achieve this distinction. Will you link your personal online identities (e.g.an existing Twitter username or Facebook account) with your learning blog? Will you link your professional online identity (e.g.~published online biography on your employer's web site) with your learning blog?
\item
  Objectives: List a few objectives for developing or improving your online identity.
\item
  Remember to add a category or tag to your post using the course tag: LDRS101 (This is needed to harvest links to posts from registered course blogs for the course feed.)
\end{enumerate}

Remember: You are in charge of what you post online and deciding what you would like to share for your digital identity for the purposes of this course. Don't share high risk personal details like physical address, date of birth, name of first pet etc., which may make it easier for identity thieves to appear more credible. If unsure, consult online resources for internet safety; for example Netsafe New Zealand. (replace resource)

!! Note that we will cover online identity in more depth in Unit 5.
\end{reflect}

\hypertarget{activity-blog-my-online-biography}{%
\subsection*{Activity: Blog: My Online Biography}\label{activity-blog-my-online-biography}}
\addcontentsline{toc}{subsection}{Activity: Blog: My Online Biography}

\begin{reflect}
In this challenge you are asked to build or update your professional online biography and the ``About'' page of your course blog.

\begin{enumerate}
\def\labelenumi{\arabic{enumi}.}
\tightlist
\item
  Reflect on the following online personas, target audience and how this will impact on the style and voice of the communication medium.
\end{enumerate}

\begin{longtable}[]{@{}ll@{}}
\toprule()
\textbf{Persona} & \textbf{Primary audience} \\
\midrule()
\endhead
Personal & Friends and family \\
Professional & Employers and professional network \\
Academic & Peer learning network \\
\bottomrule()
\end{longtable}

\begin{enumerate}
\def\labelenumi{\arabic{enumi}.}
\setcounter{enumi}{1}
\item
  Choose the most appropriate medium for each of your online personas, for example:\\
  \textbar{} \textbf{Persona} \textbar{} \textbf{Medium example} \textbar{}
  \textbar--------------\textbar---------------------------------------\textbar{}
  \textbar{} Personal \textbar{} \href{https://www.facebook.com/}{Facebook} \textbar{}
  \textbar{} Professional \textbar{} \href{https://www.linkedin.com/}{Linkedin} \textbar{}
  \textbar{} Academic \textbar{} Course blog or website \textbar{}
\item
  Identify one or two professionals from your field of interest who maintain an active web presence and contribute regularly via social media. Explore their respective websites and professional listings as examples.\\
\end{enumerate}

\begin{itemize}
\tightlist
\item
  Twitter is a good place to search for individuals using popular hashtags from your field or area of study, for example ``\#highereducation''\\
\item
  Click through to their respective twitter user page. If they have a personal website listed on the user page, visit the site and review their ``About'' page.\\
\item
  Visit their employer's page and try to locate their biography on the employer's website.\\
\item
  Search for the user on Linkedin\\
\item
  Compare the user information on these different sites. Observe how they link to social media accounts, and vary the style and content presented for the different personas.
\end{itemize}

\begin{enumerate}
\def\labelenumi{\arabic{enumi}.}
\setcounter{enumi}{3}
\item
  Create or update your professional profile on Linkedin. (Consult the help page for support.)
\item
  Create or update your ``About'' page on your course blog. You may prefer using a more informal style for this page aligned with your own personality and interests. Include links to your professional profile and respective links to social media that you use.
\item
  Visit the profile pages of your active social media accounts. Update if necessary providing links back to your main page (for example, the ``About'' page on your course site).
\item
  Think carefully about information you post publicly and keep a clear distinction between your personal online presence and your professional online persona. Review your privacy settings on your personal account(s).
\end{enumerate}
\end{reflect}

\hypertarget{evaluating-online-tools}{%
\section{Evaluating Online Tools}\label{evaluating-online-tools}}

In this topic, we invite you to think critically about the tools you use online. More than the obvious questions about whether the tool works well, is easy to navigate, etc., how do we evaluate digital tools, platforms, and interactions based on \emph{ethical principles}?

\ldots{}

Improve your skills in becoming a critical consumer of web tools
Publish a critical product review for an online tool of your choice.

\hypertarget{activity-8}{%
\subsection*{Activity:}\label{activity-8}}
\addcontentsline{toc}{subsection}{Activity:}

\begin{reflect}
Read the following articles about \ldots{}

\url{https://www.internetsociety.org/policybriefs/privacy/}

\url{https://ethicaledtech.info/wiki/Meta:About}

\href{https://www.wikihow.com/Write-a-Product-Review}{How to write a product review}
\href{https://tosdr.org/}{Terms of service. Didn't read}

{Questions to Consider}

After completing the activities above, answer the following questions:

\begin{itemize}
\tightlist
\item
  \ldots{}
\end{itemize}
\end{reflect}

\hypertarget{activity-9}{%
\subsection*{Activity:}\label{activity-9}}
\addcontentsline{toc}{subsection}{Activity:}

\begin{reflect}
In this challenge, you are invited to critically evaluate an online tool. Please select any online tool, or choose one from the list below.

Examples of software as service

\begin{itemize}
\tightlist
\item
  Blogging: Blogger, WordPress, Medium, Tumblr
\item
  File sharing: Dropbox, Nextcloud, MediaFire, Google Drive, SugarSync
\item
  Presentations: Haikudeck, Prezi, Google Slides, Slides (using Reveal.js)
\item
  Online collaboration: Basecamp, Slack, Rocket.chat, Hipchat
\item
  Video conferencing: jitsi, Anymeeting, Zoom, GoToMeeting
\item
  Feed aggregators: Feedly, Panada, NewsBlur, Inoreader, Feedreader.
\item
  Project management: Trello, Kanboard, Freedcamp, Asana, Notion, GitHub
\end{itemize}

What tool did you select? Let us know by posting a comment on Discourse sharing the name of the tool you selected and why.

Note: Your comment will be displayed in the course feed.

{Questions to Consider}

After completing the activities above, answer the following questions:

\begin{itemize}
\tightlist
\item
  \ldots{}
\end{itemize}
\end{reflect}

\hypertarget{summary-3}{%
\section*{Summary}\label{summary-3}}
\addcontentsline{toc}{section}{Summary}

In this unit, you have had the opportunity to learn about \ldots{}

\hypertarget{assessment-3}{%
\section*{Assessment}\label{assessment-3}}
\addcontentsline{toc}{section}{Assessment}

\begin{assessment}
{Digital Literacy Portfolio}

The learning activities in this unit were designed to support you as you build your \emph{Digital Literacy Portfolio}.

Your portfolio will demonstrate how you apply digital tools to support your learning. It measures the following course learning outcomes:\\
- Create a personalized narrative to document and express your learning process\\
- Evaluate digital tools, platforms, and interactions based on ethical principles\\
- Practice evaluative judgment to document your process of learning in complex domains of knowledge

\textbf{Performance Indicators}\\
\emph{I can do the following:}\\
- Maintain a public personal blog as my elearning portfolio.\\
- Interact constructively with public online learning communities using forums and social media.\\
- Annotate and discuss web resources publicly in support of my learning.\\
- Share with my learning peer group recommendations for online resources that I have found to be useful.\\
- Use citation management software for my personal online resource library.

See the Assessment section in Moodle for instructions, including the grading rubric.
\end{assessment}

\hypertarget{checking-your-learning-3}{%
\section*{Checking your Learning}\label{checking-your-learning-3}}
\addcontentsline{toc}{section}{Checking your Learning}

\begin{progress}
Before you move on to the next unit, check that you are able to:

\begin{itemize}
\tightlist
\item
  \ldots{}
\item
  \ldots{}
\end{itemize}
\end{progress}

\hypertarget{building-a-network-of-people}{%
\chapter{Building a Network of People}\label{building-a-network-of-people}}

\hypertarget{overview-4}{%
\section*{Overview}\label{overview-4}}
\addcontentsline{toc}{section}{Overview}

\ldots{}

\hypertarget{topics-4}{%
\subsection*{Topics}\label{topics-4}}
\addcontentsline{toc}{subsection}{Topics}

This unit is divided into the following topics:

\begin{enumerate}
\def\labelenumi{\arabic{enumi}.}
\tightlist
\item
  Digital Citizenship
\item
  Online Identity for Learning
\item
  Digital Environments
\end{enumerate}

\hypertarget{learning-outcomes-4}{%
\subsection*{Learning Outcomes}\label{learning-outcomes-4}}
\addcontentsline{toc}{subsection}{Learning Outcomes}

When you have completed this unit, you should be able to:

\begin{itemize}
\tightlist
\item
  \ldots{}
\item
  \ldots{}
\end{itemize}

\hypertarget{activity-checklist-4}{%
\subsection*{Activity Checklist}\label{activity-checklist-4}}
\addcontentsline{toc}{subsection}{Activity Checklist}

Here is a checklist of learning activities you will benefit from in completing this unit. You may find it useful for planning your work.

\begin{reflect}
{Learning Activities}

\begin{itemize}
\tightlist
\item
  \ldots{}
\item
  \ldots{}
\end{itemize}

\textbf{Note:} The learning activities in this course are designed to prepare you for the graded assigments in this course.\\
You are strongly encouraged to complete them.
\end{reflect}

\begin{assessment}
{Assessment}

\begin{itemize}
\tightlist
\item
  See the Assessment section in Moodle for assignment details and due dates.
\end{itemize}
\end{assessment}

\hypertarget{resources-4}{%
\subsection*{Resources}\label{resources-4}}
\addcontentsline{toc}{subsection}{Resources}

\begin{itemize}
\tightlist
\item
  All resources will be provided online in the unit.
\end{itemize}

\hypertarget{topic-6}{%
\section{Topic}\label{topic-6}}

content

\hypertarget{activity-10}{%
\subsection*{Activity:}\label{activity-10}}
\addcontentsline{toc}{subsection}{Activity:}

\begin{reflect}
\begin{itemize}
\tightlist
\item
  \href{https://educationtechnologysolutions.com/2014/07/why-you-need-a-personal-learning-network/}{Why you need a Personal Learning Network by Education Technology Solutions}.
\end{itemize}

{Questions to Consider}

After completing the activities above, answer the following questions:

\begin{itemize}
\tightlist
\item
  \ldots{}
\end{itemize}
\end{reflect}

\hypertarget{topic-7}{%
\section{Topic}\label{topic-7}}

Social media for learning

Exploring how social media can support online learning and implications for engagement and identity

The purpose of this mini challenge is to:

Review social media technologies and how they can support or inhibit learning in a digital age.
Reflect on engagement in online communities.
Consider the relationship between sharing learning online and your digital footprint and online identity.

\begin{center}\rule{0.5\linewidth}{0.5pt}\end{center}

Annotation - Learners on the periphery

Read: Using social media for learning. A guide to becoming strategic published by Sheffield Hallam University.
Explore: The conversation prism developed by Brian Solis.
Reflect on what social media technologies you use for learning and how this impacts on your digital footprint and online identity.
Share what social media technologies you use to support learning and how you use them by posting a WENote, for example:

I use to
In the future, I plan to use to

\begin{center}\rule{0.5\linewidth}{0.5pt}\end{center}

Frameworks for online engagement

Read the following article and add or reply to annotations using on the Hypothes.is focusing on how the research might apply to your own behaviour. Remember to tag your posts using the course code: LiDA102. (Consult the OERu support site for help on using the Hypothes.is annotation tool.)

Honeychurch, S., Bozkurt, A., Singh, L., \& Koutropoulos, A. (2017). Learners on the Periphery: Lurkers as Invisible Learners. European Journal of Open, Distance and E-Learning, 20(1). (Direct hypothes.is link.)

Read Derek Wenmouth's blog post: Participation online -- Four C's (How does this apply to your own online engagement?).
Watch the following video based on Ross Mayfield's power law of participation. Note how low threshold engagement on social media leaves a bread crumb (digital footprint) and generates a form of collective intelligence. Higher forms of engagement result in collaborative intelligence (note that Cheryl Reynolds refers to Yammer in the video, but the framework also applies to other social media platforms). When viewing the video, think about how you engage in different online communities.

Forum - Social media, online identity and learning

Join the discussion on social media, online identity and learning by sharing your personal views and thoughts. Choose one or more of the following questions as a catalyst for your contributions to the forum:

How much of what you learn should be open and transparent (i.e.~public) and how much should be kept private? Why?
In a digital age, how important is it for you to build a digital footprint of your learning?
What are the challenges and opportunities for building your online identity?
What levels of online engagement do your feel are appropriate for your own learning on this course? Does this differ from your engagement in other online communities?
Other?
Please ``Like'', share and reply to posts. These are forms of engagement and a contribution to your online learning identity. Remember to tag your posts using the course code: LiDA102.

\hypertarget{activity-11}{%
\subsection*{Activity:}\label{activity-11}}
\addcontentsline{toc}{subsection}{Activity:}

\begin{reflect}
Watch/read\ldots{}

{Questions to Consider}

After completing the activities above, answer the following questions:

\begin{itemize}
\tightlist
\item
  \ldots{}
\end{itemize}
\end{reflect}

\hypertarget{topic-8}{%
\section{Topic}\label{topic-8}}

content

\hypertarget{activity-12}{%
\subsection*{Activity:}\label{activity-12}}
\addcontentsline{toc}{subsection}{Activity:}

\begin{reflect}
Watch/read\ldots{}

{Questions to Consider}

After completing the activities above, answer the following questions:

\begin{itemize}
\tightlist
\item
  \ldots{}
\end{itemize}
\end{reflect}

\hypertarget{summary-4}{%
\section*{Summary}\label{summary-4}}
\addcontentsline{toc}{section}{Summary}

In this unit, you have had the opportunity to learn about \ldots{}

\hypertarget{assessment-4}{%
\section*{Assessment}\label{assessment-4}}
\addcontentsline{toc}{section}{Assessment}

\begin{assessment}
{Quizzes 1 \& 2}

After completing this unit, including the learning activities, you are asked to complete \ldots{}
\end{assessment}

\hypertarget{checking-your-learning-4}{%
\section*{Checking your Learning}\label{checking-your-learning-4}}
\addcontentsline{toc}{section}{Checking your Learning}

\begin{progress}
Before you move on to the next unit, check that you are able to:

\begin{itemize}
\tightlist
\item
  \ldots{}
\item
  \ldots{}
\end{itemize}
\end{progress}

\hypertarget{sharing-your-knowledge}{%
\chapter{Sharing your Knowledge}\label{sharing-your-knowledge}}

\hypertarget{overview-5}{%
\section*{Overview}\label{overview-5}}
\addcontentsline{toc}{section}{Overview}

\ldots{}

\hypertarget{topics-5}{%
\subsection*{Topics}\label{topics-5}}
\addcontentsline{toc}{subsection}{Topics}

This unit is divided into the following topics:

\begin{enumerate}
\def\labelenumi{\arabic{enumi}.}
\tightlist
\item
  TWU Online Community
\item
  Digital Practices in the Workplace
\item
  Societal Issues and the Internet
\end{enumerate}

\hypertarget{learning-outcomes-5}{%
\subsection*{Learning Outcomes}\label{learning-outcomes-5}}
\addcontentsline{toc}{subsection}{Learning Outcomes}

When you have completed this unit, you should be able to:

\begin{itemize}
\tightlist
\item
  \ldots{}
\item
  \ldots{}
\end{itemize}

\hypertarget{activity-checklist-5}{%
\subsection*{Activity Checklist}\label{activity-checklist-5}}
\addcontentsline{toc}{subsection}{Activity Checklist}

Here is a checklist of learning activities you will benefit from in completing this unit. You may find it useful for planning your work.

\begin{reflect}
{Learning Activities}

\begin{itemize}
\tightlist
\item
  \ldots{}
\item
  \ldots{}
\end{itemize}

\textbf{Note:} The learning activities in this course are designed to prepare you for the graded assigments in this course.\\
You are strongly encouraged to complete them.
\end{reflect}

\begin{assessment}
{Assessment}

\begin{itemize}
\tightlist
\item
  See the Assessment section in Moodle for assignment details and due dates.
\end{itemize}
\end{assessment}

\hypertarget{resources-5}{%
\subsection*{Resources}\label{resources-5}}
\addcontentsline{toc}{subsection}{Resources}

\begin{itemize}
\tightlist
\item
  All resources will be provided online in the unit.
\end{itemize}

\hypertarget{topic-9}{%
\section{Topic}\label{topic-9}}

content

\hypertarget{activity-13}{%
\subsection*{Activity:}\label{activity-13}}
\addcontentsline{toc}{subsection}{Activity:}

\begin{reflect}
Watch/read\ldots{}

{Questions to Consider}

After completing the activities above, answer the following questions:

\begin{itemize}
\tightlist
\item
  \ldots{}
\end{itemize}
\end{reflect}

\hypertarget{topic-10}{%
\section{Topic}\label{topic-10}}

content

\hypertarget{activity-14}{%
\subsection*{Activity:}\label{activity-14}}
\addcontentsline{toc}{subsection}{Activity:}

\begin{reflect}
Watch/read\ldots{}

{Questions to Consider}

After completing the activities above, answer the following questions:

\begin{itemize}
\tightlist
\item
  \ldots{}
\end{itemize}
\end{reflect}

\hypertarget{topic-11}{%
\section{Topic}\label{topic-11}}

content

\hypertarget{activity-15}{%
\subsection*{Activity:}\label{activity-15}}
\addcontentsline{toc}{subsection}{Activity:}

\begin{reflect}
Watch/read\ldots{}

{Questions to Consider}

After completing the activities above, answer the following questions:

\begin{itemize}
\tightlist
\item
  \ldots{}
\end{itemize}
\end{reflect}

\hypertarget{summary-5}{%
\section*{Summary}\label{summary-5}}
\addcontentsline{toc}{section}{Summary}

In this unit, you have had the opportunity to learn about \ldots{}

\hypertarget{assessment-5}{%
\section*{Assessment}\label{assessment-5}}
\addcontentsline{toc}{section}{Assessment}

\begin{assessment}
{Quizzes 1 \& 2}

After completing this unit, including the learning activities, you are asked to complete \ldots{}
\end{assessment}

\hypertarget{checking-your-learning-5}{%
\section*{Checking your Learning}\label{checking-your-learning-5}}
\addcontentsline{toc}{section}{Checking your Learning}

\begin{progress}
Before you move on to the next unit, check that you are able to:

\begin{itemize}
\tightlist
\item
  \ldots{}
\item
  \ldots{}
\end{itemize}
\end{progress}

\hypertarget{references}{%
\chapter*{References}\label{references}}
\addcontentsline{toc}{chapter}{References}

The following are key references used in this course. \textbf{\emph{Check with your course syllabus for required readings.}}

  \bibliography{book.bib}

\end{document}

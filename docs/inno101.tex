% Options for packages loaded elsewhere
\PassOptionsToPackage{unicode}{hyperref}
\PassOptionsToPackage{hyphens}{url}
%
\documentclass[
]{book}
\usepackage{amsmath,amssymb}
\usepackage{iftex}
\ifPDFTeX
  \usepackage[T1]{fontenc}
  \usepackage[utf8]{inputenc}
  \usepackage{textcomp} % provide euro and other symbols
\else % if luatex or xetex
  \usepackage{unicode-math} % this also loads fontspec
  \defaultfontfeatures{Scale=MatchLowercase}
  \defaultfontfeatures[\rmfamily]{Ligatures=TeX,Scale=1}
\fi
\usepackage{lmodern}
\ifPDFTeX\else
  % xetex/luatex font selection
\fi
% Use upquote if available, for straight quotes in verbatim environments
\IfFileExists{upquote.sty}{\usepackage{upquote}}{}
\IfFileExists{microtype.sty}{% use microtype if available
  \usepackage[]{microtype}
  \UseMicrotypeSet[protrusion]{basicmath} % disable protrusion for tt fonts
}{}
\makeatletter
\@ifundefined{KOMAClassName}{% if non-KOMA class
  \IfFileExists{parskip.sty}{%
    \usepackage{parskip}
  }{% else
    \setlength{\parindent}{0pt}
    \setlength{\parskip}{6pt plus 2pt minus 1pt}}
}{% if KOMA class
  \KOMAoptions{parskip=half}}
\makeatother
\usepackage{xcolor}
\usepackage{longtable,booktabs,array}
\usepackage{calc} % for calculating minipage widths
% Correct order of tables after \paragraph or \subparagraph
\usepackage{etoolbox}
\makeatletter
\patchcmd\longtable{\par}{\if@noskipsec\mbox{}\fi\par}{}{}
\makeatother
% Allow footnotes in longtable head/foot
\IfFileExists{footnotehyper.sty}{\usepackage{footnotehyper}}{\usepackage{footnote}}
\makesavenoteenv{longtable}
\usepackage{graphicx}
\makeatletter
\def\maxwidth{\ifdim\Gin@nat@width>\linewidth\linewidth\else\Gin@nat@width\fi}
\def\maxheight{\ifdim\Gin@nat@height>\textheight\textheight\else\Gin@nat@height\fi}
\makeatother
% Scale images if necessary, so that they will not overflow the page
% margins by default, and it is still possible to overwrite the defaults
% using explicit options in \includegraphics[width, height, ...]{}
\setkeys{Gin}{width=\maxwidth,height=\maxheight,keepaspectratio}
% Set default figure placement to htbp
\makeatletter
\def\fps@figure{htbp}
\makeatother
\setlength{\emergencystretch}{3em} % prevent overfull lines
\providecommand{\tightlist}{%
  \setlength{\itemsep}{0pt}\setlength{\parskip}{0pt}}
\setcounter{secnumdepth}{5}
\usepackage{booktabs}
\usepackage{amsthm}
\makeatletter
\def\thm@space@setup{%
  \thm@preskip=8pt plus 2pt minus 4pt
  \thm@postskip=\thm@preskip
}
\makeatother

\usepackage{tcolorbox}


\newtcolorbox{blackbox}{
  colback=black,
  coltext=white,
  colframe=black,
  boxsep=5pt,
  arc=4pt}
\newtcolorbox{bonus}{
  colback=blue!15,
  colframe=blue!15,
  coltext=black!80,
  boxsep=5pt,
  arc=4pt}
\newtcolorbox{reflect}{
  colback=green!5,
  colframe=green!5,
  coltext=black!80,
  boxsep=5pt,
  arc=4pt}
\newtcolorbox{assessment}{
  colback=blue!5,
  colframe=blue!5,
  coltext=black!80,
  boxsep=5pt,
  arc=4pt}
\newtcolorbox{progress}{
  colback=purple!10,
  colframe=purple!10,
  coltext=black!80,
  boxsep=5pt,
  arc=4pt}
\newtcolorbox{video}{
  colback=yellow!5,
  colframe=yellow!5,
  coltext=black!80,
  boxsep=5pt,
  arc=4pt}
\newtcolorbox{caution}{
  colback=red!5,
  colframe=red!5,
  coltext=black!80,
  boxsep=5pt,
  arc=4pt}
\newtcolorbox{feedback}{
  colback=black!5,
  colframe=black!5,
  coltext=black!80,
  boxsep=5pt,
  arc=4pt}
\ifLuaTeX
  \usepackage{selnolig}  % disable illegal ligatures
\fi
\usepackage[]{natbib}
\bibliographystyle{apalike}
\IfFileExists{bookmark.sty}{\usepackage{bookmark}}{\usepackage{hyperref}}
\IfFileExists{xurl.sty}{\usepackage{xurl}}{} % add URL line breaks if available
\urlstyle{same}
\hypersetup{
  pdftitle={Innovation 101},
  pdfauthor={Course Developed by Colin Madland \& Kelly Marjanovic},
  hidelinks,
  pdfcreator={LaTeX via pandoc}}

\title{Innovation 101}
\author{Course Developed by Colin Madland \& Kelly Marjanovic}
\date{Last updated Jul 2023}

\begin{document}
\maketitle

{
\setcounter{tocdepth}{1}
\tableofcontents
}
\hypertarget{welcome}{%
\chapter*{Welcome}\label{welcome}}
\addcontentsline{toc}{chapter}{Welcome}

This is the course book for LDRS 101: Learning with Technology. This book is divided into 6 units of study to help you engage with the course learning outcomes and prepare for the course assessment.

On the page below you will find a summary of the course syllabus, as well as how to navigate this book. Please also refer the schedule in Moodle, as well as the Assessment section in Moodle for instructions on assignments.

If you have any questions, do not hesitate to ask. We are here to help and be your guide on this journey.

\begin{quote}
The syllabus includes key information about the course schedule, assignments, and policies. Please read the full course syllabus, which you will find in Moodle. For information on how to navigate through this course on Moodle, see \href{https://trinitywestern.teamdynamix.com/TDClient/1904/Portal/KB/?CategoryID=8214}{here}.
\end{quote}

\hypertarget{course-description}{%
\subsection*{Course Description}\label{course-description}}
\addcontentsline{toc}{subsection}{Course Description}

Introduces theories and competencies related to learning and thriving in a digital world. Explores how learners are situated in `the digital' throughout their lives and how they can use digital technologies to enhance and enrich their experience of learning, working, and playing. Learners will begin to build a curated digital footprint, initiate and develop personal and professional learning networks; develop competencies to allow them to evaluate and choose digital platforms and tools that are safe and ethical; and explore how to use digital technologies to discover, curate, connect, and share knowledge with their communities.

\hypertarget{meet-your-instructors}{%
\subsection*{Meet Your Instructors}\label{meet-your-instructors}}
\addcontentsline{toc}{subsection}{Meet Your Instructors}

{[}insert{]}

\hypertarget{course-notes}{%
\section*{Course Notes}\label{course-notes}}
\addcontentsline{toc}{section}{Course Notes}

\hypertarget{how-to-navigate-this-book}{%
\subsection*{How To Navigate This Book}\label{how-to-navigate-this-book}}
\addcontentsline{toc}{subsection}{How To Navigate This Book}

Take a moment to experiment with the controls in the toolbar at the top of the page. You can search this book for a word or phrase (for example, to look up a definition). To move quickly to different portions of the book, click on the appropriate chapter or section in the table of contents on the left. The buttons at the top of the page allow you to show/hide the table of contents, search the book, adjust the typeface, the font size, and the background colour to make the text easier to read.

\includegraphics{assets/course-intro/menu.png}

The faint left and right arrows at the sides of each page (or bottom of the page if it's narrow enough) allow you to step to the next/previous section. Here's what they look like:

\includegraphics{assets/course-intro/left_arrow.png} \includegraphics{assets/course-intro/right_arrow.png}

You can also download an offline copy of this books in a pdf format. If you are having any accessibility or navigation issues with this book, please reach out to your instructor or our online team at \href{mailto:elearning@twu.ca}{\nolinkurl{elearning@twu.ca}}

\hypertarget{course-units}{%
\subsection*{Course Units}\label{course-units}}
\addcontentsline{toc}{subsection}{Course Units}

This course is organized into 10 units. Each unit of the course will provide you with the following information:

\begin{itemize}
\tightlist
\item
  A general overview of the key concepts that will be addressed during the unit.
\item
  Specific learning outcomes and topics for the unit.
\item
  Learning activities to help you engage with the concepts. These often include key readings, videos, and reflective prompts.
\item
  The Assessment section provides details on assignments you will need to complete throughout the course to demonstrate your understanding of the course learning outcomes.
\end{itemize}

\begin{caution}
 Note that assessments, including assignments and discussion posts will
 be submitted in Moodle. See the Assessment tab in Moodle for assignment
 details and dropboxes.
 \end{caution}

\hypertarget{course-activities}{%
\subsection*{Course Activities}\label{course-activities}}
\addcontentsline{toc}{subsection}{Course Activities}

Below is some key information on features you will see throughout the course.

\begin{reflect}
\textbf{\emph{Learning Activity}}\\
This box will prompt you to engage in course concepts, often by viewing
resources and reflecting on your experience and/or learning. Most
learning activities are ungraded and are designed to help prepare you
for the assessment in this course.
\end{reflect}

\begin{assessment}
\textbf{\emph{Assessment}}\\
This box will signify an assignment you will submit in Moodle. Note that
assignments demonstrate your understanding of the course learning
outcomes. Be sure to review the grading rubrics for each assignment.
\end{assessment}

\begin{progress}
\textbf{\emph{Checking Your Learning}}\\
This box is for checking your understanding, to make sure you are ready
for what follows.
\end{progress}

\begin{feedback}
\textbf{\emph{Note}}\\
This box signifies key notes, important quotes, or case students. It may
also warn you of possible problems or pitfalls you may encounter!
\end{feedback}

\hypertarget{introduction-to-digital-literacies-for-online-learning}{%
\chapter{Introduction to Digital Literacies for Online Learning}\label{introduction-to-digital-literacies-for-online-learning}}

\hypertarget{overview}{%
\section*{Overview}\label{overview}}
\addcontentsline{toc}{section}{Overview}

Welcome to Unit 1 of Learning with Technology! In this first unit, we begin the course by discussing\ldots{}

\hypertarget{topics}{%
\subsection*{Topics}\label{topics}}
\addcontentsline{toc}{subsection}{Topics}

This unit is divided into the following topics:

\begin{enumerate}
\def\labelenumi{\arabic{enumi}.}
\tightlist
\item
  Understanding the Digital
\item
  Personal Learning Environments
\item
  Online Identity for Learning
\item
  Digital Literacies
\end{enumerate}

\hypertarget{learning-outcomes}{%
\subsection*{Learning Outcomes}\label{learning-outcomes}}
\addcontentsline{toc}{subsection}{Learning Outcomes}

When you have completed this unit, you should be able to:

\begin{itemize}
\tightlist
\item
  Describe your engagement with digital technology
\item
  Apply digital tools to support learning in an academic environment
\item
  Explain what digital literacies mean for you in an academic and professional context
\item
  Examine your digital footprint
\item
  Build your professional online biography
\item
  Examine privacy concerns related to various platforms and tools
\item
  Describe how to protect yourself and others in the digital environment.
\end{itemize}

\hypertarget{activity-checklist}{%
\subsection*{Activity Checklist}\label{activity-checklist}}
\addcontentsline{toc}{subsection}{Activity Checklist}

Here is a checklist of learning activities you will benefit from in completing this unit. You may find it useful for planning your work.

\begin{reflect}
{Learning Activities}

\begin{itemize}
\tightlist
\item
  Watch the introduction video on \ldots and read\ldots{}
\item
  Establish a personal learning environment by setting up your own course blog.
\item
  Ilustrate your engagement with digital technology through a mind map.
\item
  Publish your first blog post, introducing yourself to course participants.\\
\item
  Audit your own digital footprint to find out what exists on the internet about you
\item
  Share what social media technologies you use to support learning and how you use them by posting in the community forum in Discourse.
\item
  Build or update your professional online biography and the ``About'' page of your academic / course website.\\
\item
  Post in Discourse, sharing a comment about your learning on this course.
\item
  Read an online article and annotate it using Hypothes.is.
\item
  Search, evaluate, select, annotate, tag, and share resource links.
\item
  Post in the Discourse forum, reflecting on the reasons why digital literacy matters to you
\item
  Blog about your personal definition of digital literacies and the digital visitor / digital resident personal learning network (PLN) mapping exercise
\item
  Take the Self-Check Quiz on Unit 1 concepts. (ungraded)
\end{itemize}

\textbf{Note:} The learning activities in this course are designed to prepare you for the graded assigments in this course.\\
You are strongly encouraged to complete them.
\end{reflect}

\begin{assessment}
{Assessment}

\begin{itemize}
\tightlist
\item
  See the Assessment section in Moodle for assignment details and due dates.
\end{itemize}
\end{assessment}

\hypertarget{resources}{%
\subsection*{Resources}\label{resources}}
\addcontentsline{toc}{subsection}{Resources}

\begin{itemize}
\tightlist
\item
  All resources will be provided online in the unit.
\end{itemize}

\hypertarget{understanding-the-digital}{%
\section{Understanding the Digital}\label{understanding-the-digital}}

We begin Unit 1 with an introduction to \ldots{}

\hypertarget{activity}{%
\subsection*{Activity:}\label{activity}}
\addcontentsline{toc}{subsection}{Activity:}

\begin{reflect}
Watch the video below to get an overview of \ldots.

Next, read the article on\ldots{}

{Questions to Consider}

After completing the activities above, answer the following questions:

\begin{itemize}
\tightlist
\item
  Why is it important to \ldots{}
\end{itemize}
\end{reflect}

\hypertarget{activity-reflective-journal}{%
\subsection*{Activity: Reflective Journal}\label{activity-reflective-journal}}
\addcontentsline{toc}{subsection}{Activity: Reflective Journal}

\begin{reflect}
\textbf{\emph{Feel free to answer the questions above in your notes or Reflective Learning Journal}}.

{Introduction to the Reflective Journaling}

A reflective journal is simply a record of your thoughts. It is a reflection of the way you think and the manner in which you respond to your learning. Journals can consist of traditional note taking, mind maps, pictures, stream-of-consciousness writing, recordings, quotes, sketches, or drawings: whatever you choose to include. Experiment and have fun.
The purpose of journaling is to make you an active participant in your learning experiences as you engage in the various activities throughout the course's readings, activities, and discussions with your instructor and your fellow students. Reflecting upon these learning events will help you gain a deeper understanding of the course materials and help integrate your learning into applied practice in your everyday life and work.
Throughout the course, we will remind you to write in your journal, as we want to be sure you are actively learning the material. To assist you, we have provided you with questions you can ask yourself in order to get your creative energies flowing. Reflective journaling is an activity you can and should complete on a regular or daily basis, even outside of our scheduled course activities.

\textbf{Common Questions Used for Reflective Journaling} - Click to expand

\begin{itemize}
\tightlist
\item
  In your view, what were the most important points in the readings, videoclips, or discussions with your peers?\\
\item
  What information did you already know?\\
\item
  What new knowledge, ideas, or perspectives have you gained?\\
\item
  What information was easy to remember or learn? Why?\\
\item
  What concepts did you find more difficult? Why?\\
\item
  How can you apply this knowledge to your work or current experience?\\
\item
  How has this knowledge helped you to make sense of your current or previous experience?\\
\item
  Has your understanding of a personal or work-related situation changed after studying these concepts?\\
\item
  Did you agree or disagree with any of the material? If yes, how did you react and why?\\
\item
  If you could have the opportunity to engage in further learning, what would it be?\\
\item
  What further questions would like to ask the author of your readings?\\
\item
  What other articles, books or discussions would be of interest?
\end{itemize}

\emph{For any journal assignments in your courses, please refer to the specific questions and/or grading
criteria to help you in your writing.}
\end{reflect}

\hypertarget{personal-learning-environments}{%
\section{Personal Learning Environments}\label{personal-learning-environments}}

\ldots{}

The primary purpose of this topic is to enable you to set up your course blog, which will form the hub of your Personal Learning Environment (PLE).

Blog posts are a useful way of reflecting on your learning and a means to network with your peers.
It also provides our learning community with a way to see how you are getting on and to help where we can.

The purpose of a PLE is to put the learner at the centre of the online learning environment, which will be enabled by establishing a personal blog for this course. Using a course blog:

You will retain control of your data and learning outputs generated during this online course, even after the course is completed.
You get to choose:
- The blog service you would like to use, although we recommend WordPress as it is supported by TWU.
- Whether to accept comments on your blog from your peers
- Whether to register your blog for the aggregated course feed so that any posts tagged with the course code (LiDA101) will be harvested for the feed.

A key teaching philosophy of this course is to embed the acquisition of new digital literacies
into your learning journey. Knowledge of how to use the Internet and social media technologies
will better prepare you for life in a digital world.

If this is your first time blogging, you should spend time in setting up your personal digital
learning environment.

In this course, you will also use social media technologies to interact with fellow students.
The advantage is that you will control and retain access to all the content and learning artefacts
you create during this course, even after its conclusion.

Please remember that your course blog and the social media technologies you use on this course
are public, and that you take full responsibility for anything you publish.
Do not disclose any confidential information and respect the privacy of your employer, colleagues
and friends. In short, don't say anything that you would not want to read on the Internet.

\hypertarget{activity-1}{%
\subsection*{Activity: \ldots{}}\label{activity-1}}
\addcontentsline{toc}{subsection}{Activity: \ldots{}}

\begin{reflect}
Download and read \href{https://library.educause.edu/resources/2009/5/7-things-you-should-know-about-personal-learning-environments}{``7 Things you should know about personal learning environments''}.
Read \href{https://educationtechnologysolutions.com/2014/07/why-you-need-a-personal-learning-network/}{Why you need a Personal Learning Network by Education Technology Solutions}.

{Questions to Consider}

After watching the video, consider the following:

\begin{itemize}
\tightlist
\item
  \ldots{}
\end{itemize}
\end{reflect}

\hypertarget{activity-setting-up-your-course-blog}{%
\subsection*{Activity: Setting up your course blog}\label{activity-setting-up-your-course-blog}}
\addcontentsline{toc}{subsection}{Activity: Setting up your course blog}

\begin{reflect}
As this is a course focusing on digital literacies, you are asked to establish
a course blog, as this will improve your skills and enable you to network
with your peers.

We recommend using WordPress, as it is supported by TWU.

Follow the steps below to set up your blog.

\begin{enumerate}
\def\labelenumi{\arabic{enumi}.}
\tightlist
\item
  Register an account and create a free blog on WordPress.com.
\item
  Determine the difference between the dashboard used for editing and the published view of your blog. (It is important to know the difference because, when you register your blog for the course feed, you must use the url for the public view of your blog).
\end{enumerate}

\begin{itemize}
\tightlist
\item
  Do you know how to open the published (public view) of your blog in a new window?
\item
  Have you added a browser bookmark to your dashboard and public version of your blog?
\end{itemize}

\begin{enumerate}
\def\labelenumi{\arabic{enumi}.}
\tightlist
\item
  Complete your personal details for display on the ``About'' page of your blog.
\end{enumerate}

\begin{itemize}
\tightlist
\item
  Can you see the updates on your ``about'' page in the published view of your blog?
\end{itemize}

\begin{enumerate}
\def\labelenumi{\arabic{enumi}.}
\tightlist
\item
  Review and customise your blog settings from the dashboard according to your preferences.
\end{enumerate}

\begin{itemize}
\tightlist
\item
  We recommend that you enable categories and tags on your blog.
\item
  Consider whether you want to moderate all comments or, in the case of WordPress, you can also enable previously approved posters to post comments without moderation. Conduct a web search on ``moderating blog comments'' to find out more about the pros and cons on the flow of information when moderating comments.
\end{itemize}

\begin{enumerate}
\def\labelenumi{\arabic{enumi}.}
\tightlist
\item
  Visit the appearance option on your dashboard and personalise your blog by:
\end{enumerate}

\begin{itemize}
\tightlist
\item
  Changing: your theme, header image, background colours and/or image
\item
  Add at least one widget to your blog --- remember ``less is more``. One or two of the following are functional choices: Archives, recent posts, categories or category cloud, and blogs I follow.
\end{itemize}

\begin{enumerate}
\def\labelenumi{\arabic{enumi}.}
\tightlist
\item
  Draft a blog post, reflecting on your experience of this e-Learning activity on creating a blog.
  Click on ``save draft'' (so you can review before publishing live on the web). Your reflection could for example:
\end{enumerate}

\begin{itemize}
\tightlist
\item
  Introduce yourself and reflect on what you would like to achieve by maintaining a blog to support your learning
\item
  Reflect on what you thought of the activity; Was it easy or hard?
\item
  Share links to any additional resources you found useful in completing the tasks.
\item
  Provide tips for future learners who will be completing this activity. If you were to set up a new blog again, what would you do differently?
\item
  Add anything your readers may find interesting or useful.
\end{itemize}

\begin{enumerate}
\def\labelenumi{\arabic{enumi}.}
\tightlist
\item
  Choose a photo to add to your blog post.
\end{enumerate}

\begin{itemize}
\tightlist
\item
  Be sure you have permission to upload the photo. We sugest using an open licence site, such as pixabay or\ldots{}
\item
  Alternatively, upload your own photo, take a selfie or ask someone to take a photo of you working on this blog post challenge.
\item
  Optional: Record a short video introduction and embed this in your blog post.
\end{itemize}

\begin{enumerate}
\def\labelenumi{\arabic{enumi}.}
\item
  Review your draft post and, when you're happy with what you've written, click on the ``Publish post'' button.
\item
  Add a category or tag for your post using the course tag: LDRS101
\item
  Post in the LDRS 101 Discourse forum to let your peers know the web address of your blog and ask them to post a comment. This will give you the opportunity to experience how comments function on your blog and to test if they are working properly.
\end{enumerate}
\end{reflect}

\hypertarget{online-identity-for-learning}{%
\section{Online Identity for Learning}\label{online-identity-for-learning}}

add intro paragraph

\href{https://www.internetsociety.org/learning/digital-footprints/}{Your digital footprint matters}, published by the Internet Society.

\url{https://www.youtube.com/watch?v=Ro_LlRg8rGg\&t=1s} - Four Reasons to Care About Your Digital Footprint

Julia Angwin, author of ``Dragnet Nation: A Quest for Privacy, Security, and Freedom in a World of Relentless Surveillance'' attempted to erase her digital footprint.
Watch this short CBS This Morning interview with Julia:

\ldots{}

\hypertarget{activity-digital-footprint-audit}{%
\subsection*{Activity: Digital Footprint Audit}\label{activity-digital-footprint-audit}}
\addcontentsline{toc}{subsection}{Activity: Digital Footprint Audit}

\begin{reflect}
In this activity you will audit your own digital footprint in order to find out what exists on the internet about you, and reflect on what you want your online identity to be. Follow the steps below to begin.

\begin{enumerate}
\def\labelenumi{\arabic{enumi}.}
\tightlist
\item
  Conduct a Google search of your own name (using an incognito or private window in Chrome or Firefox). Search for your first name and surname without parenthesis (for example: snow white) and then with parenthesis (for example: ``snow white''). Explore the results of your search.
\item
  Conduct a Google search of your name with the name of current and previous employers.
\item
  Conduct a Google search of your name with the name of previous schools you attended.
\item
  Expand your search to include social media sites, for example: ``snow white'' twitter; ``snow white'' facebook; ``snow white'' youtube etc.
\item
  Note any interesting or surprising findings.
\end{enumerate}

{Questions to Consider}

After watching the video, consider the following:

\begin{itemize}
\tightlist
\item
  \ldots{}
\end{itemize}
\end{reflect}

\hypertarget{activity-blog-my-digital-footprint}{%
\subsection*{Activity: Blog: My Digital Footprint}\label{activity-blog-my-digital-footprint}}
\addcontentsline{toc}{subsection}{Activity: Blog: My Digital Footprint}

\begin{reflect}
Prepare and publish a short blog post of about 250 to 300 words focusing on what you hope to achieve with your online digital identity for learning. Your post can include:

\begin{enumerate}
\def\labelenumi{\arabic{enumi}.}
\item
  Optional reflection: You may want to include a reflection(s) on the outcomes of your footprint audit. Remember that your blog post is public, so only share what you are comfortable sharing with the world. You don't need to be specific; for example, you can generalise: I am satisfied with my digital footprint because \ldots{} or I would like to improve my digital footprint for learning because \ldots{}
\item
  Professional versus private: Consider how you want to separate your ``private'' online identity from your professional and / or learning identity. If you already maintain an online presence (existing blog, twitter and/or facebook accounts) think about how you will separate professional / learning posts from private and social life interactions online. For example, maintaining a separate course or learning blog is one way to achieve this distinction. Will you link your personal online identities (e.g.an existing Twitter username or Facebook account) with your learning blog? Will you link your professional online identity (e.g.~published online biography on your employer's web site) with your learning blog?
\item
  Objectives: List a few objectives for developing or improving your online identity.
\item
  Remember to add a category or tag to your post using the course tag: LDRS101 (This is needed to harvest links to posts from registered course blogs for the course feed.)
\end{enumerate}

Remember: You are in charge of what you post online and deciding what you would like to share for your digital identity for the purposes of this course. Don't share high risk personal details like physical address, date of birth, name of first pet etc., which may make it easier for identity thieves to appear more credible. If unsure, consult online resources for internet safety; for example Netsafe New Zealand. (replace resource)

!! Note that we will cover online identity in more depth in Unit 5.
\end{reflect}

\hypertarget{digital-literacies}{%
\section{Digital Literacies}\label{digital-literacies}}

In this unit we've started discussing `the digital'; you have started a blog (part of your PLN), and you've examined your own digital identity. In our final topic, we want to begin to define digital literacies, and continue to build digital skills.

\hypertarget{activity-2}{%
\subsection*{Activity: \ldots{}}\label{activity-2}}
\addcontentsline{toc}{subsection}{Activity: \ldots{}}

\begin{reflect}
\ldots{}

{Questions to Consider}

After watching the video, consider the following:

\begin{itemize}
\tightlist
\item
  \ldots{}
\end{itemize}
\end{reflect}

\hypertarget{summary}{%
\section*{Summary}\label{summary}}
\addcontentsline{toc}{section}{Summary}

In this first unit, you have had the opportunity to learn about \ldots{}

\hypertarget{assessment}{%
\section*{Assessment}\label{assessment}}
\addcontentsline{toc}{section}{Assessment}

\begin{assessment}
{Quizzes 1 \& 2}

After completing this unit, including the learning activities, you are asked to complete \ldots{}
\end{assessment}

\hypertarget{checking-your-learning}{%
\section*{Checking your Learning}\label{checking-your-learning}}
\addcontentsline{toc}{section}{Checking your Learning}

\begin{progress}
Before you move on to the next unit, check that you are able to:

\begin{itemize}
\tightlist
\item
  Describe your engagement with digital technology
\item
  Apply digital tools to support learning in an academic environment
\item
  Explain what digital literacies mean for you in a tertiary education context
\item
  Examine your digital footprint
\item
  Build your professional online biography
\item
  Examine privacy concerns related to various platforms and tools
\item
  Describe how to protect yourself, other students and colleagues, to stay safe in the digital environment.
\end{itemize}
\end{progress}

This is text Word To define this is more text

\hypertarget{title}{%
\chapter{Title}\label{title}}

\hypertarget{title-1}{%
\chapter{Title}\label{title-1}}

\hypertarget{title-2}{%
\chapter{Title}\label{title-2}}

\hypertarget{title-3}{%
\chapter{Title}\label{title-3}}

\hypertarget{title-4}{%
\chapter{Title}\label{title-4}}

\hypertarget{title-5}{%
\chapter{Title}\label{title-5}}

\hypertarget{title-6}{%
\chapter{Title}\label{title-6}}

\hypertarget{references}{%
\chapter*{References}\label{references}}
\addcontentsline{toc}{chapter}{References}

The following are key references used in this course. \textbf{\emph{Check with your course syllabus for required readings.}}

  \bibliography{book.bib}

\end{document}
